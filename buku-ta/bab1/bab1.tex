\vspace{0ex}
\chapter {PENDAHULUAN}

Pada bab ini, akan dijelaskan mengenai latar belakang, rumusan masalah, batasan masalah, tujuan, metodologi pengerjaan, dan sistematika penulisan Tugas Akhir.

\section{Latar Belakang}

\par \textit{Computational geometry} adalah cabang dari ilmu komputer yang dikhususkan untuk mempelajari algoritma yang dapat dinyatakan dalam suatu geometri. Salah satu algoritma yang sering dipakai pada \CG adalah algoritma \CH. \textit{Convex hull} adalah sebuah set poligon dari titik pada bidang Euclidean atau ruang Euclidean, atau dapat disebut himpunan cembung terkecil yang berisi titik. Sebagai contoh, ketika suatu kumpulan titik merupakan bagian yang dibatasi dalam sebuah bidang, \CH dapat divisualisasikan sebagai bentuk yang tertutup oleh karet gelang yang membentang di sekitar titik - titik tersebut. Berikut merupakan contoh dari \CH :
\begin{figure}[!h]
	\Centering
	\includegraphics [width=\textwidth]{bab1/img/ilustrasi-convex-hull}
	\caption {Ilustrasi Convex Hull}
	\label {fig:ilustrasi-convex-hull}
\end{figure}
\par \textit{Relative convex hull} merupakan penurunan dari \textit{convex hull}. \textit{Relative convex hull} merupakan \textit{convex hull} yang mempunyai \textit{cavity} (cekungan kedalam) yang diakibatkan atau relatif terhadap sesuatu yang membatasi \textit{convex hull} tersebut. ilustrasi \textit{relative convex hull} dapat dilihat pada gambar \ref{fig:ilustrasi-relative-convex-hull}.

\begin{figure}[!h]
	\Centering
	\includegraphics [width=0.5\columnwidth]{bab2/img/ilustrasi-relative-convex-hull}
	\caption {Ilustrasi Relative Convex Hull}
	\label {fig:ilustrasi-relative-convex-hull}
\end{figure}

\par Pada topik Tugas Akhir ini akan dijelaskan algoritma penyelesaian untuk mencari \textit{relative convex hull} dari sekumpulan titik yang berada di dalam sebuah polygon sederhana dengan menggunakan algoritma \textit{Melkman Convex Hull} yang dimodifikasi dengan studi kasus pada Sphere Online Judge 5637 LL and ErBao.

\section {Rumusan Masalah}

Rumusan masalah yang diangkat dalam Tugas Akhir ini adalah sebagai berikut :

\begin {enumerate}
    \item Bagaimana mencari \RCH dari kumpulan titik di dalam sebuah poligon?
    \item Bagaimana algoritma \textit{Melkman convex hull} yang dimodifikasi menyelesaikan masalah \RCH dari kumpulan titik?
\end {enumerate}

\section {Batasan Masalah}
\label{sec:batasan_masalah}
Permalsahan yang dibahas pada Tugas Akhir ini memiliki beberapa batasan, yaitu sebagai berikut :
\begin{enumerate}
    \item Implementasi algoritma \textit{Melkman convex hull} yang di modifikasi sebagai penyelesaian permasalahan \RCH pada soal ISUN1.
    \item Algoritma \RCH terbatas pada analisis intuitif yang logis.
\end{enumerate}
Berikut merupakan batasan pada situs Sphere Online Judge: 
\begin {enumerate}
    \item Implementasi dilakukan menggunakan bahasa pemrograman C++.
    \item Banyaknya sisi pada poligon pembatas ($n$) diantara 3 sampai 500.
    \item banyaknya pohon yang berada dalam taman ($m$) diantara 0 sampai 500.
    \item Batas maksimum untuk tiap vertex memenuhi($ x $, $ y $) dimana nilai |$x$|, |$y$| $\leq 10000$. 
    \item Banyak soal tidak diketahui karena program berhenti sampai EOF.
    \item Batas waktu yang diberikan adalah $ 0.142 $ detik.
    \item Batas memori yang diberikan adalah $ 1.536 $ MB.
    \item Batas kode sumber yang diberikan adalah $ 50.000 $ B. 
\end {enumerate}

\section {Tujuan}
Tujuan Tugas Akhir ini adalah sebagai berikut :
\begin{enumerate}
    \item Mengevaluasi kinerja algoritma \textit{Melkman Convex Hull} yang dimodifikasi untuk menyelesaikan permasalahan komputasi \textit{relative convex hull} pada LL and ErBao.
\end{enumerate}

\section {Manfaat}
Tugas Akhir ini mampu memberikan pemahaman algoritma yang tepat untuk menyelesaikan permasalahan komputasi \textit{relative convex hull} dengan efisien.

\section {Metodologi}
Metodologi pengerjaan yang digunakan pada Tugas Akhir ini memiliki beberapa tahapan. Tahapan-tahapan tersebut yaitu :

\begin{enumerate}
    \item Penyusunan proposal\\
    Pada tahapan ini penulis memberikan penjelasan mengenai apa yang penulis akan lakukan dan mengapa Tugas Akhir ini dilakukan. Penjelasan tersebut dituliskan dalam bentuk proposal Tugas Akhir.
    \item Studi literatur\\
    Pada tahapan ini penulis mengumpulkan referensi yang diperlukan guna mendukung pengerjaan Tugas Akhir. Referensi yang digunakan dapat berupa hasil penelitian yang sudah pernah dilakukan, buku, artikel internet, atau sumber lain yang bisa dipertanggungjawabkan.
    \item Implementasi algoritma\\
    Pada tahapan ini penulis mulai mengembangkan algoritma yang digunakan untuk menyelesaikan permasalahan komputasi \textit{relative convex hull}.
    \item Pengujian dan evaluasi\\
    Pada tahapan ini penulis menguji performa algoritma yang digunakan. Hasil pengujian kemudian dievaluasi untuk kemudian dipertimbangkan apakah algoritma masih bisa ditingkatkan lagi atau tidak.
    \item Penyusunan buku\\
    Pada tahapan ini penulis menyusun hasil pengerjaan Tugas Akhir mengikuti format penulisan Tugas Akhir.
\end{enumerate}

\section {Sistematika Penulisan}
Sistematika laporan Tugas Akhir yang akan digunakan adalah sebagai berikut :
\begin{enumerate}
    \item BAB I : PENDAHULUAN\\
    Bab ini berisi latar belakang, rumusan masalah, batasan masalah, tujuan, manfaat, metodologi dan sistematika penulisan Tugas Akhir.
    \item BAB II : DASAR TEORI\\
    Bab ini berisi dasar teori mengenai permasalahan dan algoritma penyelesaian yang digunakan dalam Tugas Akhir
    \item BAB III : DESAIN\\
    Bab ini berisi desain algoritma dan struktur data yang digunakan dalam penyelesaian permasalahan.
    \item BAB IV : IMPLEMENTASI\\
    Bab ini berisi implementasi berdasarkan desain algortima yang telah dilakukan pada tahap desain.
    \item BAB V : UJI COBA DAN EVALUASI\\
    Bab ini berisi uji coba dan evaluasi dari hasil implementasi yang telah dilakukan pada tahap implementasi.
    \item BAB VI : PENUTUP\\
    Bab ini berisi kesimpulan dan saran yang didapat dari hasil uji coba yang telah dilakukan.
\end{enumerate}