\chapter {IMPLEMENTASI}

\section{Lingkungan implementasi}

Lingkungan implementasi dan pengembangan yang dilakukan adalah sebagai berikut.
\begin{enumerate}
	\item Perangkat Keras
	\begin{enumerate}
		\item Processor Intel® Core™ i7-7500 CPU @ 3.56GHz (4 CPUs), ~3.0GHz
		\item Random Access Memory 8192MB
	\end{enumerate}
	\item Perangkat Lunak
	\begin{enumerate}
		\item Sistem Operasi Windows 10 Education 64-bit
		\item Visual Studio Code
		\item Bahasa Pemrograman C++
		\item Kompiler GCC 7.4.0 (Ubuntu 7.4.0-1ubuntu1~18.04.1) untuk Windows Subsystem Linux
	\end{enumerate}
\end{enumerate}

\section{Implementasi Program Utama}

Subbab ini menjelaskan implementasi proses algoritma secara Keseluruhan berdasarkan desain yang telah dijelaskan pada Bab \ref{sec:desain}. Program ini merupakan program yang digunakan untuk menyelesaikan permasalahan \textit{Factorial Modulo Prime}.

\subsection{Penggunaan Library, Tipe data}

Program ini menggunakan beberapa \textit{library} dan tipe data seperti yang ditunjukkan pada kode sumber \ref{code:header_main}.

\begin{code}[firstnumber=1,float]{\textit{Header} Program Utama}{header_main}
#include <cstdio>
#include <cstdlib>
#include <unordered_map>
#include <utility>
#include <cmath>
#include <vector>

typedef unsigned long long uint64;
typedef          long long int64;
\end{code}

Untuk mengakomodasi aritmatika modular, sebuah \textit{struct} dengan tujuan tersebut diimplementasikan. Kode sumber \ref{code:zelement_1} sampai \ref{code:zelement_5} menjelaskan implementasi \textit{struct} yang dimaksud.

\newpage

\begin{code}[firstnumber=0]{\protect\textit{Struct} ZElement (bg. 1)}{zelement_1}
ZElement
uint64 character;
uint64 element;
uint64 order;

ZElement (uint64 character, uint64 element, uint64 order = 0)
{
	this->character = character;
	this->element = element % character;
	this->order = (order == 0) ? character - 1 : order; // Assumes the element is primitive element
}

ZElement (const ZElement & other)
{
\end{code}
\pagebreak
\begin{code}[firstnumber=last]{\textit{Struct} ZElement (bg. 2)}{zelement_2}
	this->character = other.character;
	this->element = other.element;
	this->order = other.order;
}

ZElement ()
{   character = 0; element = 0; order = 0; }

ZElement & operator = (const uint64 & other)
{
	this->element = other % character;
	return *this;
}

ZElement operator + (const uint64 & other)
{
	uint64 output = (element + (other % character)) % character;
	return ZElement (character, output);
}

// TODO: may fail when other >= 2 * character
ZElement operator - (const uint64 & other)
{
	uint64 output = (element >= other) ?
	element - other :
	element - other + character;
	return ZElement (character, output);
}

ZElement operator * (const uint64 & other)
{
	return ZElement (character, modularMultiplication (element, other, character));
\end{code}
\pagebreak
\begin{code}[firstnumber=last]{\textit{Struct} ZElement (bg. 3)}{zelement_3}
}

ZElement operator + (const ZElement & other)
{
	return *this + other.element;
}

ZElement operator - (const ZElement & other)
{
	return *this - other.element;
}

ZElement operator * (const ZElement & other)
{
	return *this * other.element;
}

ZElement & operator += (const uint64 & other)
{
	this->element = (element + (other % character)) % character;
	return *this;
}

// TODO: may fail when other >= 2 * character
ZElement & operator -= (const uint64 & other)
{
	this->element = (element >= other) ?
	element - other :
	element - other + character;
	return *this;
}

ZElement & operator *= (const uint64 & other)
{
\end{code}
\pagebreak
\begin{code}[firstnumber=last]{\textit{Struct} ZElement (bg. 4)}{zelement_4}
	ZElement result = ZElement (character, modularMultiplication (element, other, character));
	this->element = result.element;
	return *this;
}

ZElement & operator += (const ZElement & other)
{
	*this += other.element;
	return *this;
}

// TODO: may fail when other >= 2 * character
ZElement & operator -= (const ZElement & other)
{
	*this -= other.element;
	return *this;
}

ZElement & operator *= (const ZElement & other)
{
	*this *= other.element;
	return *this;
}

bool operator < (const ZElement & other) const
{
	return element < other.element;
}

bool operator == (const ZElement & other) const
{
	return element == other.element;
}
\end{code}
\pagebreak
\begin{code}[firstnumber=last]{\textit{Struct} ZElement (bg. 5)}{zelement_5}
	
bool operator != (const ZElement & other) const
{
	return element != other.element;
}

ZElement invert ()
{
	return ZElement (character, modularInverse (element, character));
}
\end{code}

\subsection{Implementasi Fungsi Main}

Subbab ini menjabarkan implementasi fungsi \textit{main} yang dijelaskan pada subbab~\ref{sec:Desain Fungsi Main}. Implementasi fungsi ini dapat dilihat pada kode sumber \ref{code:main}.

\begin{code}[firstnumber=0]{Fungsi Main}{main}
int main()
uint64  N, // Number of bit for q
L, // Number of bit for p
q, // Global basis
p, // Public key modulus
g, // Generator
y, // Public key query i.e. moduled generator
M; // Hashed message

scanf ("%llu%llu%llu%llu%llu%llu%llu", &N, &L, &q, &p, &g, &y, &M);
uint64 index = BSGS (g, p, y, q);

auto signedPair = sign (M, q, g, p, index);

printf ("%llu %llu\n", signedPair.first.element, signedPair.second.element);
\end{code}

\subsection{Implementasi Fungsi Baby-step Giant-step}

Subbab ini menjabarkan implementasi fungsi \textit{Baby-step Giant-step} yang dijelaskan pada subbab \ref{sec:Desain Fungsi Baby-step Giant-step}. Implementasi fungsi ini dapat dilihat pada kode sumber \ref{code:bsgs_1} sampai \ref{code:bsgs_2}.

\begin{code}[firstnumber=0]{Fungsi \textit{Baby-step Giant-step} (bg. 1)}{bsgs_1}
uint64 BSGS (const uint64 & generator, const uint64 & modulo, const uint64 & query, const uint64 & order)
typedef ZElement key_t;   // Will contain g^(limit * i) mod p
typedef ZElement value_t; // Will contain limit * i mod q

std::unordered_map <key_t, value_t, ZElement_hash> giantStep;

uint64 limit = std::ceil (std::pow (order, 0.5));

ZElement giantStepSize = modularExponentiation (ZElement (modulo, generator, order), limit);
key_t   currentGiantStepKey (modulo, 1);
value_t currentGiantStepValue (order, 0);

for (uint64 i = 0; i <= limit; i++)
{
	giantStep[ currentGiantStepKey ] = currentGiantStepValue;
	currentGiantStepKey *= giantStepSize;
	currentGiantStepValue += limit;
}

key_t poweredResult (modulo, query);
for (uint64 i = 0; i < limit; i++)
{
\end{code}
\pagebreak
\begin{code}[firstnumber=last]{Fungsi \textit{Baby-step Giant-step} (bg. 2)}{bsgs_2}
	auto it = giantStep.find ( poweredResult );
	if (it != giantStep.end())
	return (it->second - i).element;
	poweredResult *= generator;
}

return -1;
\end{code}

Fungsi ini menggunakan \textit{library} \texttt{unordered\textunderscore map}. Untuk itu, dibutuhkan sebuah fungsi untuk menghasilkan \textit{hash value} dari tipe data \texttt{ZElement}. Fungsi tersebut dapat dilihat pada kode sumber \ref{code:zelement_hash}.

\begin{code}[firstnumber=0]{Penghitung Nilai \textit{Hash} Tipe Data ZElement}{zelement_hash}
struct ZElement_hash
size_t operator () (const ZElement & in) const
{
	return  std::hash <uint64> () (in.character) ^
	std::hash <uint64> () (in.element) ^
	std::hash <uint64> () (in.order);
}
\end{code}

\subsection{Implementasi Fungsi Pollard Rho}

Subbab ini akan menjelaskan bagaimana desain fungsi \textit{Pollard Rho} akan diimplementasikan. Fungsi ini merupakan sebuah modul yang komponennya sangat berkaitan. Untuk itu, segala hal yang berkaitan dengan \textit{Pollard Rho} dikumpulkan dalam sebuah \textit{struct} pada kode sumber \ref{code:general_pollard_rho}.

\begin{code}[firstnumber=0, float=h!]{Struktur \textit{Struct Pollard Rho}}{general_pollard_rho}
struct pollardRho
/// variable
// ...
// ...

/// static functions
// ...
// ...
\end{code}

\subsubsection{Penggunaan Struct Step}

Sebuah \textit{struct} untuk akan digunakan untuk melakukan proses \textit{step}. Selain itu, \textit{struct} ini juga berfungsi untuk menyimpan histori \textit{step} setiap \textit{pointer}. \textit{Struct} ini merupakan implementasi desain pada subbab \ref{sec:Desain Fungsi Step Untuk Pollard Rho}. Implementasi \textit{struct} ini dapat dilihat pada kode sumber \ref{code:step_1} dan \ref{code:step_2}.

\begin{code}[firstnumber=0, float]{Struct \textit{Step} (bg. 1)}{step_1}
struct step
ZElement value;
ZElement alpha;
ZElement beta;

step (ZElement value, ZElement alpha, ZElement beta)
{
	this->value = value;
	this->alpha = alpha;
	this->beta = beta;
}

step ()
{
	this->value = ZElement (1, 0);
	this->alpha = ZElement (1, 0);
	this->beta = ZElement (1, 0);
}

void stepPartitionOne (const uint64 & betaValue)
{
	beta += 1;
	value *= betaValue;
}

void stepPartitionTwo ()
{
	alpha += alpha;
	beta += beta;
	value *= value;
}
\end{code}
\begin{code}[firstnumber=last, float=h]{Struct \textit{Step} (bg. 2)}{step_2}
void stepPartitionThree (const uint64 & alphaValue)
{
	alpha += 1;
	value *= alphaValue;
}

/**
*  Defines the step function
*  S1 consists of residue class [1]
*  S2 consists of residue class [0]
*  S3 consists of residue class [2]
*/
step & goForward (const uint64 & A, const uint64 & B)
{
	int residue = value.element & 3;
	if      (residue == 1) stepPartitionOne(B);
	else if (residue == 0) stepPartitionTwo();
	else                   stepPartitionThree(A);
	
	return *this;
}
\end{code}

\textit{Struct} ini memiliki tiga variabel yang penting dalam melakukan proses \textit{step}. Nilai ini tertera pada tabel \ref{tab:var_step}. Nilai A dan B akan ditentukan mengikuti penjelasan pada subbab \ref{sec:Strategi Penyelesaian Logaritma Diskret dengan Pollard Rho}.

\begin{table}[h!]
\caption{Variabel Struct Step}
\label{tab:var_step}
\begin{tabularx}{\linewidth}{ |c|c|X| }
\hline
No	&Nama	&Deskripsi \\
\hline
1	&\textit{value}	&Menyimpan nilai $ v=A^{\alpha} B^{\beta}\ (mod\ m) $ \\
2	&\textit{alpha}	&Menyimpan berapa kali step ini telah dikalikan dengan A \\
3	&\textit{beta}	&Menyimpan berapa kali step ini telah dikalikan dengan B \\
\hline
\end{tabularx}
\end{table}

\subsubsection{Implementasi Fungsi Pollard Rho dengan Brent Cycle Detection}

Subbab ini menjabarkan implementasi fungsi \textit{Pollard Rho} dengan \textit{Brent Cycle Detection} yang dijelaskan pada subbab \ref{sec:Desain Fungsi Pollard Rho dengan Brent Cycle Detection}. Implementasi fungsi ini dapat dapat dilihat pada kode sumber \ref{code:brent_1} dan \ref{code:brent_2}.
\begin{code}[firstnumber=0]{Fungsi \textit{Pollard Rho} dengan \textit{Brent Cycle Detection} (bg. 1)}{brent_1}
static uint64 brent (const uint64 & generator, const uint64 & modulo, const uint64 & query, const uint64 & order)
uint64 nextCheckpoint = 1;
uint64 currentIteration = 0;

step lastCheckpointStep;
step currentStep (ZElement (modulo, 1), ZElement (order, 0), ZElement (order, 0));
\end{code}
\begin{code}[firstnumber=last]{Fungsi \textit{Pollard Rho} dengan \textit{Brent Cycle Detection} (bg. 2)}{brent_2}
ZElement betaDifference;

bool done = false;

while ( ! done )
{
	lastCheckpointStep = currentStep;
	nextCheckpoint = nextCheckpoint * 2;
	
	for (; ( !done ) && currentIteration < nextCheckpoint; currentIteration++)
	{
		currentStep.goForward(generator, query);
		if (currentStep.value == lastCheckpointStep.value)
		{
			betaDifference = lastCheckpointStep.beta - currentStep.beta;

			if (betaDifference.element == 0)
			{
				currentStep = randomizeStartingStep (generator, order, modulo, query);
			}
			else
			{
				done = true;
			}
		}
	}
}

ZElement result = betaDifference.invert();
result *= (currentStep.alpha - lastCheckpointStep.alpha);

return result.element;
\end{code}

\subsubsection{Implementasi Fungsi Pengacakan Posisi Awal}

Subbab ini menjabarkan implementasi fungsi pengacakan posisi awal yang akan digunakan apabila terjadi kasus yang menyebabkan metode \textit{Pollard Rho} gagal memberikan keluaran yang benar seperti yang dijelaskan pada subbab \ref{sec:Brent Cycle Detection}. Implemetasi fungsi ini dapat dilihat pada kode sumber \ref{code:randomize_step}
\begin{code}[firstnumber=0]{Fungsi Pengacakan Posisi Awal}{randomize_step}
static step randomizeStartingStep (const uint64 & generator, const uint64 & order, const uint64 & modulo, const uint64 & query)
uint64 upperLimit = order - 1;
ZElement alpha (order, 1 + (rand() % upperLimit));
ZElement beta (order, 1 + (rand() % upperLimit));

ZElement poweredAlpha = modularExponentiation (ZElement (modulo, generator, order), alpha.element);
ZElement poweredBeta = modularExponentiation (ZElement (modulo, query, order), beta.element);

ZElement value = poweredAlpha * poweredBeta;

return step (value, alpha, beta);
\end{code}

\subsection{Implementasi Fungsi Pemangkatan Modular}

Subbab ini menjabarkan implementasi fungsi pemangkatan modular yang dijelaskan pada subbab \ref{sec:Desain Fungsi Modular Exponentiation}. Implementasi fungsi ini dapat dilihat pada kode sumber \ref{code:modex}.
\begin{code}[firstnumber=0, float=h!]{Fungsi Pemangkatan Modular}{modex}
ZElement modularExponentiation (ZElement generator, uint64 exponent)
ZElement result (generator.character, 1);
if (exponent >= generator.order)
exponent = exponent % generator.order;

for (; exponent > 0; exponent = exponent >> 1)
{
	if (exponent & 1 == 1)
	{
		result = result * generator;
	}
	
	generator = generator * generator;
}

return result;
\end{code}

\subsection{Implementasi Fungsi Perkalian Modular dengan Logarithmic Modular Multiplication}

Subbab ini menjabarkan implementasi fungsi \textit{Logarithmic Modular Multiplication} yang dijelaskan pada subbab \ref{sec:Desain Fungsi Logarithmic Modular Multiplication}. Implementasi fungsi ini dapat dilihat pada kode sumber \ref{code:modmul}.
\begin{code}[firstnumber=0, float=h!]{Fungsi \textit{Logarithmic Modular Multiplication}}{modmul}
uint64 modularMultiplication (uint64 multiplier, uint64 multiplicand, uint64 modulo)
multiplier = multiplier % modulo;
multiplicand = multiplicand % modulo;

if ( multiplier < multiplicand )
std::swap (multiplier, multiplicand);

uint64 result = 0;

while (multiplier > 0)
{
	if ( multiplier & 1 == 1 )
	{
		result = (result + multiplicand) % modulo;
	}
	
	multiplier >>= 1;
	multiplicand = (multiplicand << 1) % modulo;
}

return result;
\end{code}

Untuk mempercepat kinerja fungsi, \textit{sentinel} perulangan yang digunakan adalah nilai terkecil antara \textit{multiplier} dan \textit{multiplicand}.

\subsection{Implementasi Fungsi Invers Modulus}

Subbab ini menjabarkan implementasi fungsi invers modulus yang dijelaskan pada subbab \ref{sec:Desain Fungsi Invers Modulus}. Implementasi fungsi ini dapat dilihat pada kode sumber \ref{code:modinv}.
\begin{code}[firstnumber=0, float=h!]{Fungsi Invers Modulus}{modinv}
uint64 modularInverse (uint64 value, uint64 modulus)
int64 valueCoeff, modulusCoeff;
extendedEuclidean (modulus, value % modulus, modulusCoeff, valueCoeff);

return (valueCoeff < 0) ? valueCoeff + (int64) modulus : valueCoeff;
\end{code}

\subsection{Implementasi Fungsi Sign}

Subbab ini menjabarkan implementasi fungsi \textit{Sign} yang dijelaskan pada subbab \ref{sec:Desain Fungsi Sign}. Implementasi fungsi ini dapat dilihat pada kode sumber \ref{code:sign}.
\begin{code}[firstnumber=0, float]{Fungsi \textit{Sign}}{sign}
std::pair <ZElement, ZElement> sign (const uint64 & message, const uint64 & globalModulo, const uint64 & generator, const uint64 & keyModulo, const uint64 & index) 
std::pair <ZElement, ZElement> output ((ZElement) {globalModulo, 0}, (ZElement) {globalModulo, 0});

ZElement temporalKeypair (globalModulo, 1);
do
{
	temporalKeypair += 1;
	
	output.first = modularExponentiation (ZElement (keyModulo, generator, globalModulo), temporalKeypair.element);
	output.first = ZElement (globalModulo, output.first.element);
	
	if (output.first.element == 0)
	{
		continue;
	}
	
	ZElement inverseTemporalKeypair = temporalKeypair.invert();
	ZElement transformedMessage = ZElement (globalModulo, message + (output.first * index).element);
	
	output.second = inverseTemporalKeypair * transformedMessage;
	
	if (output.second.element == 0)
	{
		output.first.element = 0;
		continue;
	}
} while (output.first.element == 0);

return output;
\end{code}

\subsection{Implementasi Fungsi Uji Kebenaran}

Subbab ini menjabarkan implementasi fungsi uji kebenaran yang dijelaskan pada subbab \ref{sec:Desain Fungsi Uji Kebenaran}. Implementasi fungsi ini dapat dilihat pada kode sumber \ref{code:verify}.
\begin{code}[firstnumber=0]{Fungsi Uji Kebenaran}{verify}
bool verify (const uint64 & globalModulo, const uint64 & keyModulo, const uint64 & generator, const uint64 & query, const uint64 & message, std::pair <ZElement, ZElement> signature)
ZElement w = signature.second.invert();
ZElement u1 = w * message;
ZElement u2 = w * signature.first;
ZElement poweredG = modularExponentiation (ZElement (keyModulo, generator, keyModulo - 1), u1.element);
ZElement poweredY = modularExponentiation (ZElement (keyModulo, query, keyModulo - 1), u2.element);
ZElement v (globalModulo, (poweredG * poweredY).element);
return v.element == signature.first.element;
\end{code}

\section{Implementasi Program Pembuat Data Uji}

Subbab ini akan menjelaskan mengenai implementasi program pembuatan data uji, mengikuti penjelasan pada subbab \ref{sec:Desain Program Pembuat Data Uji}.

\subsection{Penggunaan Library, Konstanta, dan Struct}

Program ini menggunakan \textit{library} bawaan C++, yaitu \texttt{cstdio} yang digunakan untuk menerima dan mengeluarkan hasil luaran program, \texttt{cstdlib} yang digunakan untuk melakukan randomisasi, \texttt{ctime} sebagai \textit{library} untuk melakukan \textit{random seeding}, dan \texttt{utility} yang digunakan untuk melakukan penukaran dua variabel. Selain \textit{library}, sebuah tipe data \textit{unsigned integer} 64-bit juga didefinisikan dengan nama \texttt{uint64}. Keseluruhan penggunaan \textit{library} dan konstanta dapat dilihat pada kode sumber \ref{code:header_generator}.
\begin{code}[firstnumber=1]{\textit{Header} Program Pembuat Data Uji}{header_generator}
#include <cstdio>
#include <cstdlib>
#include <ctime>
#include <utility>

#define uint64 unsigned long long
\end{code}

\subsection{Implementasi Fungsi Main}

Fungsi ini merupakan inti seluruh program. Fungsi ini tidak menerima masukan. Masukan akan diberikan secara \textit{hard-coded}. Masukan tersebut diantaranya adalah nilai $ N $ , nilai $ L $, dan banyaknya pasangan data menurut masukan $ N $ dan $ L $. Pada kode sumber berikut ini, nilai $ N $ adalah 7, nilai $ L $ adalah 10 hingga 60, dan banyaknya pasangan data menurut masukan $ N $ dan $ L $ adalah 7. Kode sumber yang dimaksud dapat dilihat pada kode sumber \ref{code:main_generator}
\begin{code}[firstnumber=0]{Fungsi \textit{Main} Program Pembuat Data Uji}{main_generator}
int main()
srand (time(NULL));
for (int i = 10; i <= 60; i++)
{
	printf ("[N = %d, L = %d]\n", 7, i);
	generateKeypair(7, i, 7);
}

return 0;
\end{code}

\subsection{Implementasi Fungsi Perkalian Modular}

Fungsi ini merupakan implementasi desain fungsi \textit{Logarithmic Modular Multiplication} pada subbab \ref{sec:Desain Fungsi Logarithmic Modular Multiplication}. Implementasi fungsi ini dapat dilihat pada kode sumber \ref{code:modmul_generator}
\begin{code}[firstnumber=0]{Fungsi \textit{Logarithmic Modular Multiplication}}{modmul_generator}
uint64 modularMultiplication (const uint64 & cMultiplicand, const uint64 & cMultiplier, const uint64 & modulus)
uint64 multiplicand = cMultiplicand % modulus;
uint64 multiplier = cMultiplier % modulus;

if ( multiplier < multiplicand )
std::swap (multiplier, multiplicand);

uint64 result = 0;

while ( multiplier > 0 )
{
	if ( multiplier & 1 == 1 )
	{
		result = (result + (multiplicand % modulus)) % modulus;
	}
	
	multiplier >>= 1;
	multiplicand = (multiplicand << 1) % modulus;
}

return result;
\end{code}

\subsection{Implementasi Fungsi Pemangkatan Modular}

Fungsi ini merupakan implementasi desain pada subbab \ref{sec:Desain Fungsi Modular Exponentiation}. Terdapat sedikit penyederhanaan pada parameter masukan, yaitu hilangnya parameter \textit{order}. Fungsi ini mengasumsikan \textit{order} sebuah elemen adalah $ modulo - 1 $. Implementasi fungsi ini dapat dilihat pada kode sumber \ref{code:modex_generator}.

\pagebreak

\begin{code}[firstnumber=0]{Fungsi Pemangkatan Modular}{modex_generator}
uint64 modularExponentiation (uint64 modulo, uint64 generator, uint64 index)
uint64 result = 1;
index = index % (modulo - 1);   //  Fermat's little theorem

for (uint64 basis = generator; index > 0; index = index >> 1)
{
	if (index & 1 == 1)
	result = modularMultiplication (result, basis, modulo);
	basis = modularMultiplication (basis, basis, modulo);
}

return result;
\end{code}

\subsection{Implementasi Fungsi Pengecekan Keprimaan}

Fungsi ini mengikuti desain pada subbab \ref{sec:Desain Fungsi Pengecekan Keprimaan}. Implementasi fungsi ini dapat dilihat pada kode sumber \ref{code:primality_test_1} dan \ref{code:primality_test_2}.
\begin{code}[firstnumber=0]{Fungsi Pengecekan Keprimaan (bg. 1)}{primality_test_1}
bool isPrime (uint64 value)
if ( value < 2 ) return false;
if ( value == 2 ) return true;
if ( value % 2 == 0 ) return false;

static uint64 primeLimit[12] = {2047, 1373653, 25326001, 3215031751, 2152302898747, 3474749660383, 341550071728321, 341550071728321, 3825123056546413051, 3825123056546413051, 3825123056546413051};

int limit;
\end{code}
\pagebreak
\begin{code}[firstnumber=last, float=h!]{Fungsi Pengecekan Keprimaan (bg. 2)}{primality_test_2}
for (limit = 0; limit < 12; limit++)
	if (value < primeLimit[limit])
		break;
uint64 s = 0;
uint64 r = value - 1;
while ( (r % 2) == 0 )
{
	s++;
	r = r >> 1;
}

static uint64 setA[12] = {2, 3, 5, 7, 11, 13, 17, 19, 23, 29, 31, 37};

for (int i = 0; i < limit; i++)
{
	uint64 a = setA[i];
	uint64 y = modularExponentiation(value, a, r);
	
	if ( (y != 1) && (y != (value - 1)) )
	{
		for (uint64 j = 0; j < s - 1 && y != value - 1; j++)
		{			
			y = modularMultiplication(y, y, value);
			if ( y == 1 ) return false;
		}
		if ( y != (value-1) ) return false;
	}
}

return true;
\end{code}

\subsection{Implementasi Fungsi Polinomial Pembuat Bilangan Prima}

Fungsi ini merupakan implementasi desain pada subbab \ref{sec:Desain Fungsi Polinomial Prima}. Implementasi fungsi ini dapat dilihat pada kode sumber \ref{code:prime_poly}.
\begin{code}[firstnumber=0, float=h!]{Fungsi Polinomial Pembuat Bilangan Prima}{prime_poly}
uint64 primePolynomial (uint64 n)
return n * n + n + 41;
\end{code}

\subsection{Implementasi Fungsi Pembuat Data Uji}

Fungsi ini merupakan implementasi desain pada subbab \ref{sec:Desain Fungsi Generate}. Terdapat sedikit penyesuaian di implementasi ini, yaitu parameter masukan $ Hm $ kini merupakan variabel yang nilainya dimasukkan secara \textit{hard-code} karena nilai ini tidak banyak berpengaruh pada proses pembuatan data uji. Selain itu, inkrementasi nilai $ p $ dilakukan dengan $ q $ untuk menjaga nilai $ p $ selalu habis dibagi dengan $ q $. Implementasi fungsi ini dapat dilihat pada kode sumber \ref{code:generator_1} dan \ref{code:generator_2}.
\begin{code}[firstnumber=0]{Fungsi Pembuat Data Uji (bg. 1)}{generator_1}
void generateKeypair (int N, int L, int limit)
uint64 Hm = 101;
uint64 NLimit = 1ULL << N;
uint64 LLimit = 1ULL << L;
uint64 lowerNLimit = 1ULL << (N - 1);
uint64 lowerLLimit = 1ULL << (L - 1);
uint64 q = 0;
int i = std::floor (std::pow (lowerNLimit, 0.5));
for (; limit > 0 && q < NLimit; )
{
	q = primePolynomial(i++);
\end{code}
\pagebreak
\begin{code}[firstnumber=last]{Fungsi Pembuat Data Uji (bg. 2)}{generator_2}
	
	if ( q < lowerNLimit || ! isPrime (q) ) continue;
	
	uint64 p = lowerLLimit;
	for (p += q - (p % q); p < LLimit && limit > 0; p += q)
	{
		if ( ! isPrime (p + 1)) continue;
		
		uint64 g = 1;
		bool solved = false;
		for (uint64 h = 2; h < q; h += 1)
		{
			g = modularExponentiation (p + 1, h, p / q);
			if ( g != 1 && modularExponentiation (p + 1, g, q) == 1 )
			{
				solved = true;
				break;
			}
		}
		
		if ( g == 1 || !solved ) continue;
		
		uint64 x = 1 + rand() % (q - 1);
		uint64 y = modularExponentiation (p + 1, g, x);
		
		printf ("%d %d %llu %llu %llu %llu %llu [%llu]\n", N, L, q, p + 1, g, y, Hm, x);
		limit--;
	}
}
\end{code}