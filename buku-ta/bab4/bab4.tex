\chapter {IMPLEMENTASI}

\section{ Lingkungan implementasi}

Lingkungan implementasi dan pengembangan yang dilakukan adalah sebagai berikut.
\begin{enumerate}
	\item Perangkat Keras
	\begin{enumerate}
		\item Processor Intel® Core™ i7-6500U CPU @ 2.50GHz (4 CPUs), ~2.6GHz
		\item Random Access Memory 8192MB
	\end{enumerate}
	\item Perangkat Lunak
	\begin{enumerate}
		\item Sistem Operasi Windows 10 Home Single Language 64-bit
		\item Visual Studio Code
		\item Bahasa Pemrograman C++
		\item Kompiler GCC 7.4.0 (Ubuntu 7.4.0-1ubuntu1~18.04.1) untuk Windows Subsystem Linux
	\end{enumerate}
\end{enumerate}

\section{ Implementasi Program Utama}

Subbab ini menjelaskan implementasi proses algoritma secara keseluruhan berdasarkan desain yang telah dijelaskan pada bab \ref{sec:desain}. Program ini merupakan program yang digunakan untuk menyelesaikan permasalahan LL and ErBao.

\subsection{ Header yang diperlukan}
Implementasi algoritma ini membutuhkan enam buah \textit{header} yaitu \texttt{iostream}, \texttt{vector}, \texttt{cmath}, \texttt{map}, \texttt{queue}, dan \texttt{algorithm} seperti yang terlihat pada kode sumber \ref{code:header_main}.

\newpage
\begin{code}[firstnumber=1]{\textit{Header} yang Diperlukan}{header_main}
#pragma GCC optimize("O3")
#pragma GCC target("avx")

#include <iostream>
#include<vector>
#include<cmath>
#include<map>
#include<queue>
#include<algorithm>
\end{code}

Selain header, terdapat juga preprocessor \textit{pragma}, digunakan untuk mengganti flag kompiler yang digunakan pada daring SPOJ.
\textit{Header} \texttt{iostream} berisi fungsi standar input output operasi yang digunakan oleh bahasa C++. \textit{Header} \texttt{vector} berisi struktur data yang digunakan untuk menyimpan data array secara dinamis. \textit{Header} \texttt{cmath} berisi fungsi-fungsi untuk operasi matematika seperti fungsi \texttt{hypot}. \textit{Header} \texttt{map} berisi struktur data untuk menyimpan data \textit{key value}. \textit{Header} \texttt{queue} berisi struktur data yang digunakan untuk menyimpan antrian data. \textit{Header} \texttt{algorithm} berisi modul yang memiliki fungsi-fungsi yang sangat berguna dalam membantu mengimplementasi algoritma yang telah dibangun, contohnya adalah fungsi \textit{reverse} dan \textit{sort}.

\subsection{ Preprocessor}
Preprocessor seperti \texttt{using} digunakan untuk membuat alias dari tipe data sesungguhnya. Terdapat empat alias yang digunakan yaitu \texttt{push\_back(x)} sebagai \texttt{pb(x)}, \texttt{pop\_back(x)} sebagai \texttt{pob(x)}, \texttt{getchar(x)} sebagai \texttt{gc(x)}, dan \texttt{for (int i = 0; i < n; i++)} sebagai \texttt{FOR(i,n)}. Preprocessor dapat dilihat pada kode sumber \ref{code:preprocessor_main}.

\newpage
\begin{code}[firstnumber=1]{\textit{Preprocessor} yang Diperlukan}{preprocessor_main}
#define pb(x) push_back(x)
#define pob(x) pop_back(x)
#define FOR(i, n) for (int i = 0; i < n; i++)
#define gc(x) getchar(x)
using namespace std;
\end{code}

\subsection{ Variabel Global}
Variabel global digunakan untuk memudahkan dalam mengakses data yang digunakan lintas fungsi/struct. kode sumber implementasi variabel global dapat dilihat pada kode sumber \ref{code:var_glob_main}. Variabel tersebut didefinisikan secara global agar dapat digunakan pada setiap fungsi.

\begin{code}[firstnumber=1]{Variabel Global yang Didefinisikan Untuk Program}{var_glob_main}
const double EPS = 0.0;
const double INF = 1E9;
map<point, int> pool;
\end{code}

\subsection{ Implementasi Fungsi Main}
Fungsi main adalah implementasi algoritma yang dirancang pada pseudocode \ref{psdo:fungsi-main}. Implementasi fungsi main dapat dilihat pada kode sumber \ref{code:main}. 
\newpage
\begin{code}[firstnumber=1]{Fungsi Main}{main}
int main(){
	int kase = 1;
	int n, m;
	while (cin >> n){
		pool.clear();
		cin >> m;
		polygon perimeter;
		vector<point> trees;

		for (int i = 0; i < n; i++){
			double a = readint(), b = readint();
			pool[point(a, b, false)]++;
			perimeter.P.push_back(point(a, b, false));
		}

		if (m == 0 || m == 1){
			printf("Case #%d: %.3lf\n", kase++, 0.0);
			continue;
		}

		for (int i = 0; i < m; i++){
			double a = readint(), b = readint();
			pool[point(a, b, true)]++;
			trees.push_back(point(a, b, true));
		}

		polygon hasil = solve(perimeter, trees);

		printf("Case #%d: %.3lf\n", kase++, hasil.perimeter());
	}
}
\end{code}

\subsection{ Implementasi Class Point}
Pada subbab ini akan dijelaskan mengenai implementasi dari class \texttt{Point} pada subbab \ref{sec:point} dan pseudocode \ref{psdo:class-point}. Implementasi dari class \texttt{Point} dapat dilihat pada kode sumber \ref{code:class-point}.

\begin{code}[firstnumber=1]{Struct Point}{class-point}
struct point{
	double x, y;
	bool fixed;
	point(){
		x = y = 0.0;
		fixed = 0;
	}
	point(double _x, double _y, bool _fixed = false){
		x = _x;
		y = _y;
		fixed = _fixed;
	}
	bool operator<(point other) const{
		if (y < other.y + EPS)
			return true;
		if (y + EPS > other.y)
			return false;
		return x < other.x + EPS;
	}
	bool operator==(point other) const{
		return same_d(x, other.x) && same_d(y, other.y);
	}
};
\end{code}

\subsection{ Implementasi Class Vec}
Pada subbab ini akan dijelaskan mengenai implementasi dari class \texttt{Vec} pada subbab \ref{sec:vec} dan pseudocode \ref{psdo:class-vec}. Implementasi dari class \texttt{Vec} dapat dilihat pada kode sumber \ref{code:class-vec}.

\newpage
\begin{code}[firstnumber=1]{Struct Vec}{class-vec}
struct vec{
	double x, y;
	vec(){
		x = y = 0.0;
	}
	vec(double _x, double _y){
		x = _x;
		y = _y;
	}
	vec(point A){
		x = A.x;
		y = A.y;
	}
	vec(point A, point B){
		x = B.x - A.x;
		y = B.y - A.y;
	}
};
\end{code}

\subsection{ Implementasi Class Line}
Pada subbab ini akan dijelaskan mengenai implementasi dari class \texttt{Line} pada subbab \ref{sec:line} dan pseudocode \ref{psdo:class-line}. Implementasi dari class \texttt{Line} dapat dilihat pada kode sumber \ref{code:class-line}.

\newpage
\begin{code}[firstnumber=1]{Struct Line}{class-line}
struct line{
	double a, b, c;
	line(){
		a = b = c = 0.0;
	}
	line(double _a, double _b, double _c){
		a = _a;
		b = _b;
		c = _c;
	}
	line(point P1, point P2){
		if (P2 < P1){
			point T;
			T = P1;
			P1 = P2;
			P2 = T;
		}
		if (same_d(P1.x, P2.x))
			a = 1.0, b = 0.0, c = -P1.x;
		else
			a = -(P1.y - P2.y) / (P1.x - P2.x), b = 1.0, c = -(a * P1.x) - P1.y;
	}
	line(point P, double slope){
		if (same_d(slope, INF))
			a = 1.0, b = 0.0, c = -P.x;
		else
			a = -slope, b = 1.0, c = -(a * P.x) - P.y;
	}
	bool operator==(line other) const{
		return same_d(a, other.a) && same_d(b, other.b) && same_d(c, other.c);
	}
};
\end{code}

\subsection{ Implementasi Class Segment}
Pada subbab ini akan dijelaskan mengenai implementasi dari class \texttt{Segment} pada subbab \ref{sec:segment} dan pseudocode \ref{psdo:class-segment}. Implementasi dari class \texttt{Segment} dapat dilihat pada kode sumber \ref{code:class-segment}.

\begin{code}[firstnumber=1]{Struct Segment}{class-segment}
struct segment{
	point P, Q;
	line L;
	segment(){
		point T1;
		P = Q = T1;
		line T2;
		L = T2;
	}
	segment(point _P, point _Q){
		if (_Q < _P){
			point T1 = _P;
			_P = _Q;
			_Q = T1;
		}
		P = _P;
		Q = _Q;
		line T2(_P, _Q);
		L = T2;
	}
	bool operator==(segment other) const{
		return P == other.P && Q == other.Q;
	}
};
\end{code}

\subsection{ Implementasi Class Polygon}
Pada subbab ini akan dijelaskan mengenai implementasi dari class \texttt{Polygon} pada subbab \ref{sec:polygon} dan pseudocode \ref{psdo:class-polygon}. Implementasi dari class \texttt{Polygon} dapat dilihat pada kode sumber \ref{code:class-polygon}.
\newpage
\begin{code}[firstnumber=1]{Struct Polygon}{class-polygon}
struct polygon{
	vector<point> P;
	polygon(){
		P.clear();
	}
	polygon(vector<point> &_P){
		P = _P;
	}
	int prev(int idx){
		return (idx == 0 ? P.size() - 1 : idx - 1);
	}
	int next(int idx){
		return (idx == P.size() - 1 ? 0 : idx + 1);
	}
	double perimeter(){
		double ret = 0;
		FOR(i, P.size()){
			ret += e_dist(P[i], P[next(i)]);
		}
		return ret;
	}
};
\end{code}

\subsection{ Implementasi Fungsi BetweenD}
Pada subbab ini akan dijelaskan mengenai implementasi dari fungsi \texttt{BetweenD} pada pseudocode \ref{psdo:fungsi-betweend}. Implementasi dapat dilihat pada kode sumber \ref{code:fungsi-betweend}.

\begin{code}[firstnumber=1]{Fungsi BetweenD}{fungsi-betweend}
double between_d(double x, double l, double r){
	return (min(l, r) <= x + EPS && x <= max(l, r) + EPS);
}
\end{code}

\subsection{ Implementasi Fungsi EDist}
Pada subbab ini akan dijelaskan mengenai implementasi dari fungsi \texttt{EDist} pada pseudocode \ref{psdo:fungsi-edist}. Implementasi dapat dilihat pada kode sumber \ref{code:fungsi-edist}.

\begin{code}[firstnumber=1]{Fungsi EDist}{fungsi-edist}
double e_dist(point P1, point P2){
	return hypot(P1.x - P2.x, P1.y - P2.y);
}
\end{code}

\subsection{ Implementasi Fungsi Cross}
Pada subbab ini akan dijelaskan mengenai implementasi dari fungsi \texttt{Cross} pada pseudocode \ref{psdo:fungsi-cross}. Implementasi dapat dilihat pada kode sumber \ref{code:fungsi-cross}.

\begin{code}[firstnumber=1]{Fungsi Cross}{fungsi-cross}
double cross(vec u, vec v){
	return (u.x * v.y - u.y * v.x);
}
\end{code}

\subsection{ Implementasi Fungsi Orientation}
Pada subbab ini akan dijelaskan mengenai implementasi dari fungsi \texttt{Orientation} pada pseudocode \ref{psdo:fungsi-orientation}. Implementasi dapat dilihat pada kode sumber \ref{code:fungsi-orientation}.

\begin{code}[firstnumber=1]{Fungsi Orientation}{fungsi-orientation}
double orientation(point O, point P, point Q){
	vec OP(O, P), OQ(O, Q);
	double c = cross(OP, OQ);
	return c;
}
\end{code}

\subsection{ Implementasi Fungsi OnSegment}
Pada subbab ini akan dijelaskan mengenai implementasi dari fungsi \texttt{OnSegment} pada pseudocode \ref{psdo:fungsi-onsegment}. Implementasi dapat dilihat pada kode sumber \ref{code:fungsi-onsegment}.

\begin{code}[firstnumber=1]{Fungsi OnSegment}{fungsi-onsegment}
bool onSegment(point P, segment S){
	if (orientation(S.P, S.Q, P) != 0.0)
		return false;
	return between_d(P.x, S.P.x, S.Q.x) && between_d(P.y, S.P.y, S.Q.y);
}
\end{code}

\subsection{ Implementasi Fungsi ConvexHull}
Pada subbab ini akan dijelaskan mengenai implementasi dari fungsi \texttt{ConvexHull} pada pseudocode \ref{psdo:fungsi-convexhull}. Implementasi dapat dilihat pada kode sumber \ref{code:fungsi-convexhull}.

\begin{code}[firstnumber=1]{Fungsi ConvexHull}{fungsi-convexhull}
polygon convexHull(vector<point> &pts){
	sort(pts.begin(), pts.end());
	vector<point> hull;
	for (int i = 0; i < 2; i++){
		int start = (int)hull.size();
		for (int i = 0; i < pts.size(); i++){
			while ((int)hull.size() >= start + 2 && orientation(hull[(int)hull.size() - 1], hull[(int)hull.size() - 2], pts[i]) <= 0.0)
				hull.pob();
			hull.pb(pts[i]);
		}
		hull.pop_back();
		reverse(pts.begin(), pts.end());
	}
	return polygon(hull);
}
\end{code}

\subsection{ Implementasi Fungsi InSimplePolygon}
Pada subbab ini akan dijelaskan mengenai implementasi dari fungsi \texttt{inSimplePolygon} pada pseudocode \ref{psdo:fungsi-insimplepolygon}. Implementasi dapat dilihat pada kode sumber \ref{code:fungsi-insimplepolygon}.

\begin{code}[firstnumber=1]{Fungsi InSimplePolygon}{fungsi-insimplepolygon}
int inSimplePolygon(point P, polygon &A){
	int ret = 0;
	FOR(i, A.P.size()){
		if (P == A.P[i])
			return 0;
		int j = A.next(i);
		if (onSegment(P, segment(A.P[i], A.P[j])))
			return 0;
		bool below = (A.P[i].y < P.y);
		if (below != (A.P[j].y < P.y)){
			double o = orientation(P, A.P[i], A.P[j]);
			if (o == 0.0)
				return 0;
			if (below == (o > 0.0))
				ret += below ? 1 : -1;
		}
	}
	return ret == 0 ? 1 : -1;
}
\end{code}

\subsection{ Implementasi Fungsi GetBetween}
Pada subbab ini akan dijelaskan mengenai implementasi dari fungsi \texttt{GetBetween} pada pseudocode \ref{psdo:fungsi-getbetween}. Implementasi dapat dilihat pada kode sumber \ref{code:fungsi-getbetween}.

\newpage
\begin{code}[firstnumber=1]{Fungsi GetBetween}{fungsi-getbetween}
vector<point> getBetween(polygon &triangle, queue<point> q, vector<point> trees){
	vector<point> points,pts;
	while (!q.empty()){
		if (inSimplePolygon(q.front(), triangle) != 1){
			points.pb(q.front());
		}
		q.pop();
	}
	for (int i = 0; i < trees.size(); i++){
		if (inSimplePolygon(trees[i], triangle) != 1){
			points.pb(trees[i]);
		}
	}
	polygon P = convexHull(points);
	int i = 0;
	while (1){
		if (P.P[i] == triangle.P[0]){
			if (P.P[P.next(i)] == triangle.P[2]){
				i = P.prev(i);
				while (!(P.P[i] == triangle.P[2])){
					pts.pb(P.P[i]);i = P.prev(i);
				}
			}
			else{
				i = P.next(i);
				while (!(P.P[i] == triangle.P[2])){
					pts.pb(P.P[i]);i = P.next(i);
				}
			}
			break;
		}
		i = P.next(i);
	}
	return pts;
}	
\end{code}

\subsection{ Implementasi Fungsi Solve}
Pada subbab ini akan dijelaskan mengenai implementasi dari fungsi \texttt{Solve} pada pseudocode \ref{psdo:fungsi-solve}. Implementasi dapat dilihat pada kode sumber \ref{code:fungsi-solve}.

\begin{code}[firstnumber=1]{Fungsi Solve}{fungsi-solve}
polygon solve(polygon &perimeter, vector<point> &trees){
	queue<point> q;
	for (int i = 0; i < perimeter.P.size(); i++){
		q.push(perimeter.P[i]);
	}

	point bef = perimeter.P[perimeter.P.size() - 1];
	while (1){
		bool erased = 0;
		int count = q.size();
		while (count--){
			point cur = q.front();
			q.pop();
			pool[cur]--;
			if (!cur.fixed && (!find(q, cur) || cur == bef || cur == q.front()) && orientation(cur, bef, q.front()) <= 0.0){
				erased = true;
				polygon triangle;
				triangle.P.pb(bef);
				triangle.P.pb(cur);
				triangle.P.pb(q.front());
				vector<point> points = getBetween(triangle, q, trees);
				for (int i = 0; i < points.size(); i++){
					q.push(points[i]);
					pool[points[i]]++;
					bef = points[i];
				}
			}
			else{
				q.push(cur);
				pool[cur]++;
				bef = cur;
			}
		}
		if (!erased)
			break;
	}

	vector<point> hull;
	while (!q.empty()){
		hull.pb(q.front());
		q.pop();
	}
	return polygon(hull);
}
\end{code}