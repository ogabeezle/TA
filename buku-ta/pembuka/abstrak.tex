\chapter {ABSTRAK}

% ---- Indonesian vers.

\noindent\textbf{\MakeUppercase\judul}
\vspace*{1em}

\begin{tabularx}{\linewidth}{ l l p{2.2in} }
	Nama 			& : & \penulis \\
	NRP 			& :	& \nrp \\
	Departemen 		& : & \jurusan, \newline \fakultas, ITS \\
	Pembimbing I 	& : & \pembimbingsatu \\
	Pembimbing II 	& : & \pembimbingdua
	\vspace*{1em} 	% HACKY--USE ALTERNATIVE IF POSSIBLE %
\end {tabularx}

\noindent\textbf{Abstrak} \\
\itshape
Faktorial sangat penting dalam perhitungan kombinatorik, untuk contoh, terdapat n! cara berbeda untuk menyusun n bilangan berbeda dalam sebuah urutan. Kemudian banyak cara untuk memilih k bilangan dari n bilangan yang diberikan yang disebut koefisien binomial, memerlukan faktorial dalam perhitungannya. Selain kombinatorik, faktorial juga berperan dalam kalkulus (Taylor’s theorem), teori probabilitas maupun dalam teori keamanan jaringan yang menggunakan domain prime field. Kemudian timbul pertanyaan, bagaimana cara mengkomputasi faktorial dengan efisien?. Topik Tugas Akhir ini mengulas dua algoritma yang digunakan untuk menyelesaikan permasalahan komputasi faktorial dalam prime field dengan efisien, yaitu menggunakan Multipoint Evaluation dan Shifting Evaluation Values. Melalui pengujian dan studi kasus, didapat hasil bahwa Shifting Evaluation Values memiliki performa yang lebih baik dari Multipoint Evaluation.

\vspace*{1em}
\noindent\bfseries Kata Kunci: geometri; convex hull; melkman; teori bilangan;
\normalfont
\cleardoublepage

% ---- English vers.
\chapter {ABSTRACT}
\noindent\textbf{\MakeUppercase\juduleng}
\vspace*{1em}

\begin{tabularx}{\linewidth}{ l l p{2.2in} }
	Name 			& : & \penulis \\
	Student ID		& :	& \nrp \\
	Department 		& : & \jurusaneng, \newline \fakultaseng, ITS \\
	Supervisor I 	& : & \pembimbingsatu \\
	Supervisor II 	& : & \pembimbingdua
	\vspace*{1em} 	% HACKY--USE ALTERNATIVE IF POSSIBLE %
\end {tabularx}
	
\noindent\textbf{Abstract} \\
\itshape
Factorial is extremely important in combinatoric calculation, for example, there are n! different ways to arrange n numbers in different order. Then how many ways to choose k number from n number given called binomial coefficients, also need factorial in computation. Besides combinatoric, factorial also has role in calculus (Taylor's theorem), probability theory and in network and security that use the prime field domain. Then, how to compute factorial efficiently?. This thesis reviews two algorithm to solve factorial computation efficiently in prime field, which is Multipoint Evaluation Algorithm and Shifting Evaluation Values. According to subsequent testing and case study, it appears that Shifting Evaluation Values have better performance compared to Multipoint Evaluation.

\vspace*{1em}
\noindent\bfseries Keywords:  multipoint evaluation; shifting evaluation values; factorial; number theory;
\normalfont
\cleardoublepage