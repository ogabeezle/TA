\chapter {ABSTRAK}

% ---- Indonesian vers.

\noindent\textbf{\MakeUppercase\judul}
\vspace*{1em}

\begin{tabularx}{\linewidth}{ l l p{2.2in} }
	Nama 			& : & \penulis \\
	NRP 			& :	& \nrp \\
	Departemen 		& : & \jurusan, \newline \fakultas, ITS \\
	Pembimbing I 	& : & \pembimbingsatu \\
	Pembimbing II 	& : & \pembimbingdua
	\vspace*{1em} 	% HACKY--USE ALTERNATIVE IF POSSIBLE %
\end {tabularx}

\noindent\textbf{Abstrak} \\
\itshape
\textit{Computational geometry} adalah cabang dari ilmu komputer yang dikhususkan untuk mempelajari algoritma yang dapat dinyatakan dalam suatu geometri. Salah satu algoritma yang sering dipakai pada \CG adalah algoritma \CH. \textit{Convex hull} adalah sebuah set polygon dari titik pada bidang \textit{euclidean} atau ruang \textit{euclidean}, atau dapat disebut himpunan cembung terkecil yang berisi titik. Convex hull dapat divisualisasikan sebagai bentuk yang tertutup oleh karet gelang yang membentang di sekitar titik - titik tersebut.\\\\
\textit{Relative convex hull} merupakan penurunan dari \textit{convex hull}. \textit{Relative convex hull} merupakan \textit{convex hull} yang mempunyai \textit{cavity} (cekungan ke dalam) yang diakibatkan atau relatif terhadap sesuatu yang membatasi \textit{convex hull} tersebut. Topik Tugas Akhir ini mengulas algoritma reduksi poligon untuk menyelesaikan permasalahan \textit{relative convex hull}. Metode penyelesaian permasalahan \textit{relative convex hull} dapat diselesaikan menggunakan metode reduksi polygon. Metode ini akan mereduksi vertex pembentuk polygon luar sehingga terbentuknya \textit{relative convex hull}.\\\\
Melalui pengujian dan studi kasus didapatkan bahwa algoritma reduksi poligon dapat menyelesaikan permasalahan \textit{relative convex hull} dengan rata-rata waktu 0.081 detik dan rata-rata memory 4.57 MB yang ekivalen dengan kompleksitas $\mathcal{O}{(NM\ \text{log}\ N)}$.

\vspace*{1em}
\noindent\bfseries Kata Kunci: geometri; convex hull; algoritma reduksi poligon; relative convex hull;
\normalfont
\cleardoublepage

% ---- English vers.
\chapter {ABSTRACT}
\noindent\textbf{\MakeUppercase\juduleng}
\vspace*{1em}

\begin{tabularx}{\linewidth}{ l l p{2.2in} }
	Name 			& : & \penulis \\
	Student ID		& :	& \nrp \\
	Department 		& : & \jurusaneng, \newline \fakultaseng, ITS \\
	Supervisor I 	& : & \pembimbingsatu \\
	Supervisor II 	& : & \pembimbingdua
	\vspace*{1em} 	% HACKY--USE ALTERNATIVE IF POSSIBLE %
\end {tabularx}
	
\noindent\textbf{Abstract} \\
\itshape
computational geometry is one of the computer science branches that mainly focus in studying geometrical algorithm. One of the algorithm that mostly used is convex hull. Convex hull is a polygon from multiple point inside of euclidean plane. In short, minimum convex polygon that covers set of points. convex hull can be visualized by a rubber band that covers the set of points.\\\\
Relative convex hull derived from convex hull. Relative convex hull is convex hull that have one or more cavity that relative to some thing that limit the convex polygon. In this Thesis will review polygon reduction algorithm to solve relative convex hull problem. Polygon reduction method can solve relative convex hull problem. This method will redice the vertex of the bounding polygon until it makes the relative convex hull. \\\\
According to sub-sequence testing and case study, it appears that polygon reduction algorithm can solve relative convex hull problem with average time 0.081 second and average memory 4.57 MB that equivalen to $\mathcal{O}{(NM\ \text{log}\ N)}$.

\vspace*{1em}
\noindent\bfseries Keywords: geometry; convex hull; polygon reduction algorithm; relative convex hull;
\normalfont
\cleardoublepage