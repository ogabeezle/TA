\chapter{KESIMPULAN}

Berdasarkan penjabaran di bab-bab sebelumnya, dapat disimpulkan beberapa poin terkait penyelesaian permasalahan DSA Attack.
\begin{enumerate}
\item Permasalahan pembuatan \textit{signature} menggunakan kunci publik tanpa mengetahui kunci privat dapat diselesaikan menggunakan \textit{Baby-step Giant-step}, perkalian modular dengan \textit{Logarithmic Modular Multiplication}, pemangkatan modular dengan \textit{Repeated Squaring}, dan pencarian invers modulus dengan \textit{Extended Euclidean} dengan kompleksitas waktu sebesar $ O(\sqrt{q} \log p) + O (\log k * \log p) + O(\log k + \log p) $.
\item Metode \textit{Pollard Rho} kurang cocok digunakan sebagai solusi penyelesaian permasalahan DSA Attack karena kemungkinan waktu yang dibutuhkan bisa sangat tinggi.
\item Penyelesaian permasalahan dengan \textit{Baby-step Giant-step} akan membutuhkan waktu yang mendekati rata-rata \textit{running time} yang dimiliki metode tersebut.
\end{enumerate}