\chapter{DESAIN}
\label{sec:desain}
Pada bab ini akan dijelaskan desain algoritma yang akan digunakan untuk menyelesaikan permasalahan.

\section{Desain Umum Sistem}
Pada subbab ini akan dijelaskan mengenai gambaran secara umum dari algoritma yang dirancang. Sistem diawali dengan menerima masukan 2 buah bilangan bulat $N$ yang merupakan banyaknya vertex pembentuk poligon luar dan $M$ yang merupakan banyaknya titik yang ada di dalam poligon tersebut. $N$ baris berikutnya berisikan 2 buah bilangan bulat $x_i$, $y_i$ yang merupakan koordinat dari vertex pembentuk poligon luar terurut berlawanan arah jarum jam. $M$ baris berikutnya berisikan dua buah bilangan bulat $x1_i$, $y1_i$ yang merupakan koordinat dari titik yang ada di dalam poligon. Pseudocode fungsi \fakesc{main} merupakan bagian fungsi utama yang menerima masukan, memproses masukan tersebut, dan menampilkan masukan tersebut. Fungsi \fakesc{Input} merupakan fungsi untuk menerima masukan, dan fungsi \fakesc{Print} merupakan fungsi untuk menampilkan hasil. 

\begin{algorithm}
	\caption{Fungsi \fakesc{main}}
	\label{psdo:function_main}
    \begin{algorithmic}[1]
        \While{($N \leftarrow $ \fakesc{Input()}) and $N \neq EOF$}
            \State $M \leftarrow$ \fakesc{Input()}
            \For {$i \leftarrow 1, N$}
                \State $ x_i , y_i \leftarrow $ \fakesc{Input()}
            \EndFor
            \For {$i \leftarrow 1, M$}
                \State $ x1_i , y1_i \leftarrow $ \fakesc{Input()}
            \EndFor
            \State $ans \leftarrow $\fakesc{solve(n,p)}
            \State \fakesc{Print} ($ans $)
        \EndWhile
	\end{algorithmic}
\end{algorithm}
\section{Desain Class Point}
\label{sec:point}
Class \fakesc{Point} adalah class untuk menyimpan titik dalam diagram Kartesius. Pseudocode \ref{psdo:class-point} merupakan pseudocode dari class \fakesc{Point}. Nantinya pada implementasi, class ini akan melakukan \textit{override} terhadap operator perbandingan.
\begin{table}[]
	\Centering
	\begin{tabular}{|c|p{7cm}|}
	\hline
	Nama Variabel & \multicolumn{1}{c|}{Fungsi Variabel}                               \\ \hline
$x$           & Menyimpan ordinat dari titik tersebut  \\ \hline
$y$           & Menyimpan absis dari titik tersebut          \\ \hline
$fixed$             & untuk membedakan antara titik pembentuk poligon $P$ dan titik yang ada di dalam kumpulan titik $S$   \\ \hline
	\end{tabular}
	\caption{Nama dan Fungsi Variabel dalam class \fakesc{Point}}
	\label{tab:var-point}
\end{table}
\begin{algorithm}
	\caption{Class \fakesc{Point}}
	\label{psdo:class-point}
	\begin{algorithmic}[1]
        \State $ x, y \leftarrow $ \textbf{double}
        \State $fixed \leftarrow $ \textbf{boolean}
		\State \textbf{constructor} \Call{\fakesc{Point}}{$ $}
        \State \textbf{constructor} \Call{\fakesc{Point}}{$ \_x, \_y, \_fixed $}
	\end{algorithmic}
\end{algorithm}

Class \fakesc{Point} tidak memiliki fungsi karena class ini memang hanya untuk menyimpan suatu titik yang akan digunakan nanti.

Fungsi \textit{Constructor} dari class ini terdiri dari dua jenis. Fungsi \textit{constructor} yang pertama adalah fungsi dengan tanpa parameter, pada \textit{constructor} ini, semua variabel yang ada di dalam class \fakesc{Point} akan di set $0$. Fungsi \textit{constructor} kedua adalah fungsi dengan parameter $\_x, \_y, \_fixed$, menyatakan nilai $x, y, fixed$ secara berurutan.

\section{Desain Class Vec}
Class \fakesc{Vec} merupakan class yang menyimpan vector dari dua buah titik pada diagram kartesian. Pseudocode \ref{psdo:class-vec} merupakan pseudocode dari class \fakesc{Vec}. 

\begin{table}[]
	\Centering
	\begin{tabular}{|c|p{7cm}|}
	\hline
	Nama Variabel & \multicolumn{1}{c|}{Fungsi Variabel}                               \\ \hline
$x$           & Menyimpan arah vektor sejajar dengan sumbu $x$  \\ \hline
$y$           & Menyimpan arah vektor sejajar dengan sumbu $y$          \\ \hline
	\end{tabular}
	\caption{Nama dan Fungsi Variabel dalam class \fakesc{Vec}}
	\label{tab:var-vec}
\end{table}
\begin{algorithm}
	\caption{Class \fakesc{Vec}}
	\label{psdo:class-vec}
	\begin{algorithmic}[1]
        \State $ x, y \leftarrow $ \textbf{double}
		\State \textbf{constructor} \Call{\fakesc{Vec}}{$ $}
        \State \textbf{constructor} \Call{\fakesc{Vec}}{$ \_x, \_y $}
        \State \textbf{constructor} \Call{\fakesc{Vec}}{$ A, B $}
	\end{algorithmic}
\end{algorithm}

Class \fakesc{Vec} tidak memiliki fungsi karena class ini hanya untuk menyimpan vector dari dua titik yang akan digunakan nanti.

Fungsi \textit{Constructor} dari class ini terdiri dari 3 jenis. Fungsi \textit{constructor} yang pertama adalah fungsi dengan tanpa parameter, pada \textit{constructor} ini, semua variabel yang ada di dalam class \fakesc{Point} akan di set $0$. Fungsi \textit{constructor} kedua adalah fungsi dengan parameter $\_x, \_y$, menyatakan nilai $x, y$ secara berurutan. Fungsi \textit{constructor} ketiga adalah fungsi dengan parameter $A, B$, menyatakan \fakesc{Point} dari titik $A$ dan \fakesc{Point} dari titik $B$, dimana nantinya nilai $x$ dan $y$ akan didapatkan dari pengurangan koordinat dari \fakesc{Point} $A$ dan \fakesc{Point} $B$.

\section{Desain Class Line}
Class \fakesc{Line} merupakan class yang bertanggung jawab untuk melakukan operasi-operasi pada garis dalam diagram kartesian. Pseudocode \ref{psdo:class-line} merupakan pseudocode dari class \fakesc{Line}. 

\begin{table}[]
	\Centering
	\begin{tabular}{|c|p{7cm}|}
	\hline
	Nama Variabel & \multicolumn{1}{c|}{Fungsi Variabel}                               \\ \hline
$a$           & Menyimpan nilai $a$ pada persamaan $ax + by + c =0$ \\ \hline
$b$           & Menyimpan nilai $b$ pada persamaan $ax + by + c =0$          \\ \hline
$c$           & Menyimpan nilai $c$ pada persamaan $ax + by + c =0$          \\ \hline
	\end{tabular}
	\caption{Nama dan Fungsi Variabel dalam class \fakesc{Line}}
	\label{tab:var-line}
\end{table}
\begin{algorithm}
	\caption{Class \fakesc{Line}}
	\label{psdo:class-line}
	\begin{algorithmic}[1]
        \State $ a, b, c \leftarrow $ \textbf{double}
		\State \textbf{constructor} \Call{\fakesc{Line}}{$ $}
        \State \textbf{constructor} \Call{\fakesc{Line}}{$ \_a, \_b, \_c $}
        \State \textbf{constructor} \Call{\fakesc{Line}}{$ A, B $}
	\end{algorithmic}
\end{algorithm}

Class \fakesc{Line} tidak memiliki fungsi karena class ini hanya untuk menyimpan nilai dari fungsi $ax+by+c=0$ yang akan digunakan nanti.

Fungsi \textit{Constructor} dari class ini terdiri dari 3 jenis. Fungsi \textit{constructor} yang pertama adalah fungsi dengan tanpa parameter, pada \textit{constructor} ini, semua variabel yang ada di dalam class \fakesc{Point} akan di set $0$. Fungsi \textit{constructor} kedua adalah fungsi dengan parameter $\_a, \_b, \_c$, menyatakan nilai $a, b, c$ secara berurutan. Fungsi \textit{constructor} ketiga adalah fungsi dengan parameter $A, B$, menyatakan \fakesc{Point} dari titik $A$ dan \fakesc{Point} dari titik $B$, dimana nantinya nilai $a$, $b$ dan $c$ akan didapatkan dengan mencari fungsi garis yang melewati \fakesc{Point} $A$ dan \fakesc{Point} $B$.