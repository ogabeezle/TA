\chapter{DESAIN}
\label{sec:desain}
Pada bab ini akan dijelaskan desain algoritma yang akan digunakan untuk menyelesaikan permasalahan.

\section{ Desain Umum Sistem}
\label{sec:desain-umum-sistem}
Pada subbab ini akan dijelaskan mengenai gambaran secara umum dari algoritma yang dirancang. Sistem diawali dengan menerima masukan 2 buah bilangan bulat $N$ yang merupakan banyaknya vertex pembentuk polygon luar dan $M$ yang merupakan banyaknya titik yang ada di dalam polygon tersebut. $N$ baris berikutnya berisikan 2 buah bilangan bulat $x_i$, $y_i$ yang merupakan koordinat dari vertex pembentuk polygon luar terurut berlawanan arah jarum jam. $M$ baris berikutnya berisikan dua buah bilangan bulat $x1_i$, $y1_i$ yang merupakan koordinat dari titik yang ada di dalam polygon.

\subsection{ Desain Fungsi Main}
Fungsi \fakesc{main} merupakan fungsi yang bertanggung jawab untuk menerima masukan yang sudah dijelaskan pada \ref{sec:desain-umum-sistem} untuk dilakukan proses selanjutnya. Pseudocode fungsi \fakesc{main} dapat dilihat pada pseudocode \ref{psdo:fungsi-main}. Fungsi \fakesc{Input} merupakan fungsi untuk menerima masukan, dan fungsi \fakesc{Print} merupakan fungsi untuk menampilkan hasil. 

\begin{algorithm}
	\caption{Fungsi \fakesc{main}}
	\label{psdo:fungsi-main}
    \begin{algorithmic}[1]
        \While{($N \leftarrow $ \fakesc{Input()}) and $N \neq EOF$}
            \State $M \leftarrow$ \fakesc{Input()}
            \State $perimeter \leftarrow$ \fakesc{Polygon}
            \State $trees \leftarrow$ Array \fakesc{Point}   
            \For {$i \leftarrow 1, N$}
                \State $x_i , y_i \leftarrow $ \fakesc{Input()}
                \State $perimeter.P[i]\leftarrow$ \fakesc{Point}($x_i$, $y_i$, false)
			\EndFor
			\If{$M=1$ or $M=0$}
				\State \fakesc{Print}($0$)
				\State \fakesc{Continue}
			\EndIf
            \For {$i \leftarrow 1, M$}
                \State $ x1_i , y1_i \leftarrow $ \fakesc{Input()}
                \State $trees \leftarrow$ \fakesc{Point}($x1_i$, $y1_i$, true)
            \EndFor
            \State $ans \leftarrow $\fakesc{Solve}($perimeter$, $trees$)
            \State \fakesc{Print} ($ans $)
        \EndWhile
	\end{algorithmic}
\end{algorithm}
\subsection{ Desain Class Point}
\label{sec:point}
Class \fakesc{Point} adalah class untuk menyimpan titik dalam diagram Kartesius. Pseudocode \ref{psdo:class-point} merupakan pseudocode dari class \fakesc{Point}. Nantinya pada implementasi, class ini akan melakukan \textit{override} terhadap operator perbandingan.
\begin{table}[htb]
	\Centering
	\caption{Nama dan Fungsi Variabel dalam Class \fakesc{Point}}
	\begin{tabular}{|c|p{7cm}|}
	\hline
	Nama Variabel & \multicolumn{1}{c|}{Fungsi Variabel}                               \\ \hline
$x$           & Menyimpan ordinat dari titik tersebut  \\ \hline
$y$           & Menyimpan absis dari titik tersebut          \\ \hline
$fixed$             & Untuk membedakan antara titik pembentuk polygon $P$ dan titik yang ada di dalam kumpulan titik $S$   \\ \hline
	\end{tabular}
	\label{tab:var-point}
\end{table}
\begin{algorithm}
	\caption{Class \fakesc{Point}}
	\label{psdo:class-point}
	\begin{algorithmic}[1]
        \State $ x, y \leftarrow $ \textbf{double}
        \State $fixed \leftarrow $ \textbf{boolean}
		\State \textbf{constructor} \Call{\fakesc{Point}}{$ $}
        \State \textbf{constructor} \Call{\fakesc{Point}}{$ \_x, \_y, \_fixed $}
	\end{algorithmic}
\end{algorithm}

Class \fakesc{Point} tidak memiliki fungsi karena class ini memang hanya untuk menyimpan suatu titik yang akan digunakan nanti.

Fungsi \textit{Constructor} dari class ini terdiri dari dua jenis. Fungsi \textit{constructor} yang pertama adalah fungsi dengan tanpa parameter, pada \textit{constructor} ini, semua variabel yang ada di dalam class \fakesc{Point} akan di inisialisasi dengan $0$. Fungsi \textit{constructor} kedua adalah fungsi dengan parameter $\_x, \_y, \_fixed$, menyatakan nilai $x, y, fixed$ secara berurutan.

\subsection{ Desain Class Vec}
\label{sec:vec}
Class \fakesc{Vec} merupakan class yang menyimpan vector dari dua buah titik pada diagram kartesian. Pseudocode \ref{psdo:class-vec} merupakan pseudocode dari class \fakesc{Vec}. 

\begin{table}[htb]
	\Centering
	\caption{Nama dan Fungsi Variabel dalam Class \fakesc{Vec}}
	\begin{tabular}{|c|p{7cm}|}
	\hline
	Nama Variabel & \multicolumn{1}{c|}{Fungsi Variabel}                               \\ \hline
$x$           & Menyimpan arah vektor absis  \\ \hline
$y$           & Menyimpan arah vektor ordinat          \\ \hline
	\end{tabular}
	\label{tab:var-vec}
\end{table}
\begin{algorithm}
	\caption{Class \fakesc{Vec}}
	\label{psdo:class-vec}
	\begin{algorithmic}[1]
        \State $ x, y \leftarrow $ \textbf{double}
		\State \textbf{constructor} \Call{\fakesc{Vec}}{$ $}
        \State \textbf{constructor} \Call{\fakesc{Vec}}{$ \_x, \_y $}
        \State \textbf{constructor} \Call{\fakesc{Vec}}{$ A, B $}
	\end{algorithmic}
\end{algorithm}

Class \fakesc{Vec} tidak memiliki fungsi karena class ini hanya untuk menyimpan vector dari dua titik yang akan digunakan nanti.

Fungsi \textit{Constructor} dari class ini terdiri dari 3 jenis. Fungsi \textit{constructor} yang pertama adalah fungsi dengan tanpa parameter, pada \textit{constructor} ini, semua variabel yang ada di dalam class \fakesc{Vec} akan di inisialisasi dengan $0$. Fungsi \textit{constructor} kedua adalah fungsi dengan parameter $\_x, \_y$, menyatakan nilai $x, y$ secara berurutan. Fungsi \textit{constructor} ketiga adalah fungsi dengan parameter $A, B$, menyatakan \fakesc{Point} dari titik $A$ dan \fakesc{Point} dari titik $B$, dimana nantinya nilai $x$ dan $y$ akan didapatkan dari pengurangan koordinat dari \fakesc{Point} $A$ dan \fakesc{Point} $B$.

\subsection{ Desain Class Line}
\label{sec:line}
Class \fakesc{Line} merupakan class yang bertanggung jawab untuk melakukan operasi-operasi pada garis dalam diagram kartesian. Pseudocode \ref{psdo:class-line} merupakan pseudocode dari Class \fakesc{Line}. 

\begin{table}[htb]
	\Centering
	\caption{Nama dan Fungsi Variabel dalam class \fakesc{Line}}
	\begin{tabular}{|c|p{7cm}|}
	\hline
	Nama Variabel & \multicolumn{1}{c|}{Fungsi Variabel}                               \\ \hline
$a$           & Menyimpan nilai $a$ pada persamaan $ax + by + c =0$ \\ \hline
$b$           & Menyimpan nilai $b$ pada persamaan $ax + by + c =0$          \\ \hline
$c$           & Menyimpan nilai $c$ pada persamaan $ax + by + c =0$          \\ \hline
	\end{tabular}
	\label{tab:var-line}
\end{table}
\begin{algorithm}
	\caption{Class \fakesc{Line}}
	\label{psdo:class-line}
	\begin{algorithmic}[1]
        \State $ a, b, c \leftarrow $ \textbf{double}
		\State \textbf{constructor} \Call{\fakesc{Line}}{$ $}
        \State \textbf{constructor} \Call{\fakesc{Line}}{$ \_a, \_b, \_c $}
        \State \textbf{constructor} \Call{\fakesc{Line}}{$ A, B $}
	\end{algorithmic}
\end{algorithm}

Class \fakesc{Line} tidak memiliki fungsi karena class ini hanya untuk menyimpan nilai dari fungsi $ax+by+c=0$ yang akan digunakan nanti.

Fungsi \textit{Constructor} dari class ini terdiri dari 3 jenis. Fungsi \textit{constructor} yang pertama adalah fungsi dengan tanpa parameter, pada \textit{constructor} ini, semua variabel yang ada di dalam class \fakesc{Line} akan di inisialisasi dengan $0$. Fungsi \textit{constructor} kedua adalah fungsi dengan parameter $\_a, \_b, \_c$, menyatakan nilai $a, b, c$ secara berurutan. Fungsi \textit{constructor} ketiga adalah fungsi dengan parameter $A, B$, menyatakan \fakesc{Point} dari titik $A$ dan \fakesc{Point} dari titik $B$, dimana nantinya nilai $a$, $b$ dan $c$ akan didapatkan dengan mencari fungsi garis yang melewati \fakesc{Point} $A$ dan \fakesc{Point} $B$.

\subsection{ Desain Class Segment}
\label{sec:segment}
Class \fakesc{Segment} merupakan class yang bertanggung jawab untuk menyimpan dan melakukan operasi-operasi pada segmen garis dalam diagram kartesian. Pseudocode \ref{psdo:class-segment} merupakan pseudocode dari class \fakesc{Segment}. 

\begin{table}[htb]
	\Centering
	\caption{Nama dan Fungsi Variabel dalam Class \fakesc{Segment}}
	\begin{tabular}{|c|p{7cm}|}
	\hline
	Nama Variabel & \multicolumn{1}{c|}{Fungsi Variabel}                               \\ \hline
$P$           & Menyimpan \fakesc{Point} yang merupakan ujung awal dari sebuah segmen garis \\ \hline
$Q$           & Menyimpan \fakesc{Point} yang merupakan ujung akhir dari sebuah segmen garis          \\ \hline
$L$           & Menyimpan fungsi dari garis yang melalui dua titik tersebut      \\ \hline
	\end{tabular}
	\label{tab:var-segment}
\end{table}
\begin{algorithm}
	\caption{Class \fakesc{Segment}}
	\label{psdo:class-segment}
	\begin{algorithmic}[1]
        \State $ P, Q \leftarrow $ \fakesc{Point}
        \State $L \leftarrow$ \fakesc{Line}
		\State \textbf{constructor} \Call{\fakesc{Segment}}{$ $}
        \State \textbf{constructor} \Call{\fakesc{Segment}}{$ \_P, \_Q$}
	\end{algorithmic}
\end{algorithm}

Class \fakesc{Segment} tidak memiliki fungsi karena class ini hanya untuk menyimpan data dari sebuah segmen garis yang akan digunakan nanti.

Fungsi \textit{Constructor} dari class ini terdiri dari 2 jenis. Fungsi \textit{constructor} yang pertama adalah fungsi dengan tanpa parameter, pada \textit{constructor} ini, semua variabel yang ada di dalam class \fakesc{Segment} akan di inisialisasi dengan $0$. Fungsi \textit{constructor} kedua adalah fungsi dengan parameter $\_P, \_Q$, menyatakan \fakesc{Point} dari titik $P$ dan \fakesc{Point} dari titik $Q$, yang merupakan titik \fakesc{Point} $A$ dan \fakesc{Point} $B$ secara berturut, dan \fakesc{Line} $L$ didapar dengan menggunakan \textit{constructor} \fakesc{Line} dengan parameter $P$ dan $Q$.

\subsection{ Desain Class Polygon}
\label{sec:polygon}
Class \fakesc{Polygon} merupakan class yang bertanggung jawab untuk menyimpan dan melakukan operasi-operasi pada polygon pada diagram kartesian. Pseudocode \ref{psdo:class-polygon} merupakan pseudocode dari class \fakesc{Polygon}. 

\begin{table}[htb]
	\Centering
	\caption{Nama dan Fungsi Variabel dalam Class \fakesc{Polygon}}
	\begin{tabular}{|c|p{7cm}|}
	\hline
	Nama Variabel & \multicolumn{1}{c|}{Fungsi Variabel}                               \\ \hline
$P$           & Menyimpan array dari \fakesc{Point} yang membentuk polygon tersebut \\ \hline
	\end{tabular}
	\label{tab:var-polygon}
\end{table}
\begin{algorithm}
	\caption{Class \fakesc{Polygon}}
	\label{psdo:class-polygon}
	\begin{algorithmic}[1]
        \State $ P \leftarrow $ Array \fakesc{Point}
		\State \textbf{constructor} \Call{\fakesc{Polygon}}{$ $}
        \State \textbf{constructor} \Call{\fakesc{Polygon}}{$ \_P$}
        \State \textbf{function} \Call{\fakesc{prev}}{$ idx $}
		\State \textbf{function} \Call{\fakesc{next}}{$ idx $}
		\State \textbf{function} \Call{\fakesc{perimeter}}{$ $}
	\end{algorithmic}
\end{algorithm}

Fungsi-fungsi yang terkandung dalam class ini adalah \fakesc{prev}, \fakesc{next}, \fakesc{perimeter}. Tabel \ref{tab:var-polygon} menjelaskan variabel dan kegunaannya dalam class \fakesc{Polygon}. 

Fungsi \textit{Constructor} dari class ini terdiri dari 2 jenis. Fungsi \textit{constructor} yang pertama adalah fungsi dengan tanpa parameter, pada \textit{constructor} ini, variabel $P$ yang ada di dalam class \fakesc{Polygon} akan di inisialisasi. Fungsi \textit{constructor} kedua adalah fungsi dengan parameter $\_P$, menyatakan array \fakesc{Point} dari titik pembentuk polygon tersebut.

Fungsi \textit{next} bertanggung jawab untuk mencari index selanjutnya dari titik yang membentuk polygon. Masukan, proses dan keluaran dari fungsi ini tercantum pada tabel \ref{tab:class-polygon-next}. Pseudocode fungsi ini dapat dilihat pada pseudocode \ref{psdo:class-polygon-next}.

\begin{table}[htb]
	\Centering
	\caption{Masukan, Proses, dan Keluaran dari Fungsi \fakesc{Next} Class \fakesc{Polygon}}
	\begin{tabular}{|p{3cm}|p{3cm}|p{3cm}|}
	\hline
	Masukan   & Proses     & Keluaran \\ \hline
	Suatu bilangan bulat $idx$ yang menyatakan index saat ini & mencari index selanjutnya &   Suatu bilangan bulat yang menyatakan index selanjutnya     \\ \hline
	\end{tabular}
	\label{tab:class-polygon-next}
\end{table}

\begin{algorithm}
    \caption{Fungsi \fakesc{Next} pada class \fakesc{Polygon}}
	\label{psdo:class-polygon-next}
	\begin{algorithmic}[1]
        \Require $ idx $
        \If{$idx = $\fakesc{Size}($P$)$-1$}
            \State \Return $0$
        \Else
            \State \Return $idx+1$
		\EndIf
	\end{algorithmic}
\end{algorithm}

Fungsi \textit{prev} bertanggung jawab untuk mencari index sebelumnya dari titik yang membentuk polygon. Masukan, proses dan keluaran dari fungsi ini tercantum pada tabel \ref{tab:class-polygon-prev}. Pseudocode fungsi ini dapat dilihat pada pseudocode \ref{psdo:class-polygon-prev}.

\begin{table}[htb]
	\Centering
	\caption{Masukan, Proses, dan Keluaran dari Fungsi \fakesc{Prev} Class \fakesc{Polygon}}
	\begin{tabular}{|p{3cm}|p{3cm}|p{3cm}|}
	\hline
	Masukan   & Proses     & Keluaran \\ \hline
	Suatu bilangan bulat $idx$ yang menyatakan index saat ini & mencari index sebelumnya &   Suatu bilangan bulat yang menyatakan index sebelumnya     \\ \hline
	\end{tabular}
	\label{tab:class-polygon-prev}
\end{table}

\begin{algorithm}
    \caption{Fungsi \fakesc{Prev} pada class \fakesc{Polygon}}
	\label{psdo:class-polygon-prev}
	\begin{algorithmic}[1]
        \Require $ idx $
        \If{$idx = 0$}
            \State \Return \fakesc{Size}($P$)$-1$
        \Else
            \State \Return $idx-1$
		\EndIf
	\end{algorithmic}
\end{algorithm}

Fungsi \textit{perimeter} bertanggung jawab untuk mencari keliling dari sebuah polygon. Masukan, proses, dan keluaran dari fungsi ini tercantum pada tabel \ref{tab:class-polygon-perimeter}. Pseudocode fungsi ini dapat dilihat pada pseudocode \ref{psdo:class-polygon-perimeter}.
\begin{table}[htb]
	\Centering
	\caption{Masukan, Proses, dan Keluaran dari Fungsi \fakesc{Perimeter} Class \fakesc{Polygon}}
	\begin{tabular}{|p{3cm}|p{3cm}|p{3cm}|}
	\hline
	Masukan   & Proses     & Keluaran \\ \hline
	- & Mencari keliling dengan mencari jarak eulerian dari semua titik pembentuk polygon &   Suatu bilangan berkoma yang menyatakan keliling dari polygon tersebut     \\ \hline
	\end{tabular}
	\label{tab:class-polygon-perimeter}
\end{table}

\begin{algorithm}
    \caption{Fungsi \fakesc{Perimeter} pada class \fakesc{Polygon}}
	\label{psdo:class-polygon-perimeter}
    \begin{algorithmic}[1]
        \State $ret \leftarrow 0$
        \For{$i \leftarrow 0$ \textbf{to} \fakesc{Size}($P$)$-1$}
            \State $ret \leftarrow ret $ \fakesc{EDist}($P[i]$, $P[$ \fakesc{next}($i$)$] $)
        \EndFor
        \State \Return $ret$
	\end{algorithmic}
\end{algorithm}

\subsection{ Desain Fungsi BetweenD}
\label{sec:fungsi-betweend}
Fungsi \fakesc{BetweenD} bertanggung jawab untuk mencari tahu apakah suatu bilangan $x$ berada diantara bilangan $l$ dan bilangan$r$. Pseudocode dari fungsi \fakesc{BetweenD} dapat dilihat pada pseudocode \ref{psdo:fungsi-betweend}. Masukan, proses, dan keluaran dari fungsi ini tercantum pada tabel \ref{tab:fungsi-betweend}.

\begin{table}[htb]
	\Centering
	\caption{Masukan, Proses, dan Keluaran dari Fungsi \fakesc{BetweenD} }
	\begin{tabular}{|p{3cm}|p{3cm}|p{3cm}|}
	\hline
	Masukan   & Proses     & Keluaran \\ \hline
	Tiga buah bilangan $x,l,r$, dimana bilangan $x$ merupakan bilangan yang ingin diketahui apakah berada diantara titik $l$ dan $r$ & Mencari tahu apakah bilangan $x$ berada diantara bilangan $l$ dan $r$ &   Sebuah boolean yang menyatakan apakah $x$ berada diantara $l$ dan $r$     \\ \hline
	\end{tabular}
	\label{tab:fungsi-betweend}
\end{table}
\begin{algorithm}
    \caption{Fungsi \fakesc{BetweenD}}
	\label{psdo:fungsi-betweend}
    \begin{algorithmic}[1]
        \Require $x, l, r$
        \If{\fakesc{Min}($l, r$) $\le x + EPS$ and $x \le$ \fakesc{Max}($l, r$) $+ EPS$}
            \State \Return \fakesc{True}
        \Else
            \State \Return \fakesc{False}
        \EndIf
	\end{algorithmic}
\end{algorithm}


\subsection{ Desain Fungsi EDist}
\label{sec:fungsi-edist}
Fungsi \fakesc{EDist} bertanggung jawab untuk mencari jarak antara dua titik \fakesc{Point} $A$ dan \fakesc{Point} $B$ dengan menggunakan rumus \textit{Pythagoras}. Rumus \textit{Pythagoras} dapat di lihat pada persamaan \ref{eq:pythagoras}. Pseudocode fungsi \fakesc{EDist} dapat dilihat pada pseudocode \ref{psdo:fungsi-edist}. Masukan, proses, dan keluaran dari fungsi ini tercantum pada tabel \ref{tab:fungsi-edist}.

\begin{equation}
    \label{eq:pythagoras}
    C=\sqrt{A^2 + B^2}
\end{equation}

\begin{table}[htb]
	\Centering
	\caption{Masukan, Proses, dan Keluaran dari Fungsi \fakesc{EDist} }
	\begin{tabular}{|p{3cm}|p{3cm}|p{3cm}|}
	\hline
	Masukan   & Proses     & Keluaran \\ \hline
	Dua buah \fakesc{Point} $A$ dan \fakesc{Point} $B$ yang akan dicari jaraknya & Mencari jarak antara \fakesc{Point} $A$ dan \fakesc{Point} $B$ &   Sebuah bilangan berkoma yang menyatakan jarak \fakesc{Point} $A$ dan \fakesc{Point} $B$  \\ \hline
	\end{tabular}
	\label{tab:fungsi-edist}
\end{table}

\begin{algorithm}
    \caption{Fungsi \fakesc{EDist}}
	\label{psdo:fungsi-edist}
    \begin{algorithmic}[1]
        \Require $A, B$
        \State \Return \fakesc{Sqrt}($(A*A)+(B*B)$)
	\end{algorithmic}
\end{algorithm}

\subsection{ Desain Fungsi Cross}
\label{sec:fungsi-cross}
Fungsi \fakesc{Cross} bertanggung jawab untuk mencari nilai perkalian \textit{cross} dari dua buah vector. Rumus Pythagoras dapat di lihat pada persamaan \ref{eq:cross}. Pseudocode fungsi \fakesc{Cross} dapat dilihat pada pseudocode \ref{psdo:fungsi-cross}. Masukan, proses, dan keluaran dari fungsi ini tercantum pada tabel \ref{tab:fungsi-cross}.

\begin{equation}
    \label{eq:cross}
    C = (u_x*v_y) - (u_y*v_x)
\end{equation}
\begin{table}[htb]
	\Centering
	\caption{Masukan, Proses, dan Keluaran dari Fungsi \fakesc{Cross} }
	\begin{tabular}{|p{3cm}|p{3cm}|p{3cm}|}
	\hline
	Masukan   & Proses     & Keluaran \\ \hline
	Dua buah \fakesc{Vec} $U$ dan \fakesc{Vec} $V$ yang akan dicari hasil perkalian silangnya & Mencari nilai perkalian silang dari \fakesc{Vec} $U$ dan \fakesc{Vec} $V$ &   Sebuah bilangan yang menyatakan hasil perkalian silang \fakesc{Vec} $U$ dan \fakesc{Vec} $V$  \\ \hline
	\end{tabular}
	\label{tab:fungsi-cross}
\end{table}


\begin{algorithm}
    \caption{Fungsi \fakesc{Cross}}
	\label{psdo:fungsi-cross}
    \begin{algorithmic}[1]
        \Require $U, V$
        \State \Return $(U.x*V.y) - (U.y*V.x)$
	\end{algorithmic}
\end{algorithm}

\subsection{ Desain Fungsi Orientation}
\label{sec:fungsi-orientation}
Fungsi \fakesc{Orientation} bertanggung jawab untuk mencari orientasi dari tiga titik. Pseudocode fungsi \fakesc{Orientation} dapat dilihat pada pseudocode \ref{psdo:fungsi-orientation}. Masukan, proses, dan keluaran dari fungsi ini tercantum pada tabel \ref{tab:fungsi-orientation}.

\begin{table}[htb]
	\Centering
	\caption{Masukan, Proses, dan Keluaran dari Fungsi \fakesc{Orientation} }
	\begin{tabular}{|p{3cm}|p{3cm}|p{3cm}|}
	\hline
	Masukan   & Proses     & Keluaran \\ \hline
	Tiga buah \fakesc{Point} $O$, \fakesc{Point} $P$ dan \fakesc{Point} $Q$ yang akan dicari orientasinya & Mencari orientasi antara \fakesc{Point} $O$, \fakesc{Point} $P$ dan \fakesc{Point} $Q$ &   Sebuah bilangan yang menyatakan orientasi antara \fakesc{Point} $O$, \fakesc{Point} $P$ dan \fakesc{Point} $Q$  \\ \hline
	\end{tabular}
	\label{tab:fungsi-orientation}
\end{table}
\begin{algorithm}
    \caption{Fungsi \fakesc{Orientation}}
	\label{psdo:fungsi-orientation}
    \begin{algorithmic}[1]
        \Require $O, P, Q$
        \State $OP \leftarrow$ \fakesc{Vec}($O,P$)
        \State $OQ \leftarrow$ \fakesc{Vec}($O,Q$)
        \State \Return \fakesc{Cross}($OP, OQ$)
	\end{algorithmic}
\end{algorithm}

\subsection{ Desain Fungsi OnSegment}
\label{sec:fungsi-onsegment}
Fungsi \fakesc{OnSegment} bertanggung jawab untuk mencari tahu apakah sebuah titik \fakesc{Point} $P$ bersentuhan dengan sebuah segmen garis \fakesc{Segment} $S$ atau tidak. Pseudocode fungsi \fakesc{OnSegment} dapat dilihat pada pseudocode \ref{psdo:fungsi-onsegment}. Masukan, proses, dan keluaran dari fungsi ini tercantum pada tabel \ref{tab:fungsi-onsegment}.

\begin{table}[htb]
	\Centering
	\caption{Masukan, Proses, dan Keluaran dari Fungsi \fakesc{OnSegment} }
	\begin{tabular}{|p{3cm}|p{3cm}|p{3cm}|}
	\hline
	Masukan   & Proses     & Keluaran \\ \hline
	Dua buah \fakesc{Point} $P$ dan \fakesc{Segment} $S$ yang akan dicari tahu apakah \fakesc{Point} $P$ berada di \fakesc{Segment} $S$ & Mencari tahu apakah \fakesc{Point} $P$ berada di \fakesc{Segment} $S$ &   Sebuah boolean yang menyatakan apakah \fakesc{Point} $P$ berada di dalam \fakesc{Segment} $S$ \\ \hline
	\end{tabular}
	\label{tab:fungsi-onsegment}
\end{table}
\begin{algorithm}
    \caption{Fungsi \fakesc{OnSegment}}
	\label{psdo:fungsi-onsegment}
    \begin{algorithmic}[1]
        \Require $P, S$
        \If{\fakesc{Orientation}($S.P, S.Q, P$)}
            \State \Return \fakesc{False}
        \Else
            \State \Return (\fakesc{BetweenD}($P.x, S.P.x, S.Q.x$) and \fakesc{BetweenD}($P.y, S.P.y, S.Q.y$))
        \EndIf
	\end{algorithmic}
\end{algorithm}

\subsection{ Desain Fungsi ConvexHull}
\label{sec:fungsi-convexhull}
Fungsi \fakesc{ConvexHull} bertanggung jawab untuk mencari \textit{convex hull} dari sekumpulan titik $pts$. Algoritma yang digunakan oleh fungsi ini adalah algoritma \textit{Monotone Chain} yang dapat dilihat pada bagian \ref{sec:algoritma-monotone-chain}. Pseudocode dari fungsi \fakesc{ConvexHull} yang digunakan dapat dilihat pada Pseudocode \ref{psdo:fungsi-convexhull}.  Masukan, proses, dan keluaran dari fungsi ini tercantum pada tabel \ref{tab:fungsi-convexhull}.

\begin{table}[htb]
	\Centering
	\caption{Masukan, Proses, dan Keluaran dari Fungsi \fakesc{ConvexHull} }
	\begin{tabular}{|p{3cm}|p{3cm}|p{3cm}|}
	\hline
	Masukan   & Proses     & Keluaran \\ \hline
    Sebuah array \fakesc{Point} $pts$ yang merupakan sekumpulan titik & Mencari \fakesc{Polygon} dengan keliling terkecil dari sekumpulan titik &   Sebuah \fakesc{Polygon} yang mengelilingi kumpulan \fakesc{Point} $pts$  \\ \hline
	\end{tabular}
	\label{tab:fungsi-convexhull}
\end{table}
\begin{algorithm}
    \caption{Fungsi \fakesc{ConvexHull}}
	\label{psdo:fungsi-convexhull}
    \begin{algorithmic}[1]
        \Require $pts$
        \State \fakesc{Sort}($pts$)
        \State $hull \leftarrow$ Array \fakesc{Point}
        \For{$i \leftarrow 0$ to $2$}
            \State $start \leftarrow$ \fakesc{Size}($hull$)
            \For{$pt$ in $pts$}
                \While{(\fakesc{Size}($hull$)$\ge start+2$) and (\fakesc{Orientation}($hull[$\fakesc{size}($hull$)$-1]$, $hull[$\fakesc{size}($hull$)$-2]$, $pt$)$\le 0$)}
                \State $hull.$\fakesc{Erase}($hull[$\fakesc{size}($hull$)$-1]$)
                \EndWhile
                \State $hull.$\fakesc{Insert}($pt$)
            \EndFor
            \State $hull.$\fakesc{Erase}($hull[$\fakesc{size}($hull$)$-1]$)
            \State \fakesc{reverse}($pts$)
        \EndFor
        \State \Return \fakesc{Polygon}($hull$);
	\end{algorithmic}
\end{algorithm}

\subsection{ Desain Fungsi InSimplePolygon}
\label{sec:fungsi-insimplepolygon}
Fungsi \fakesc{InSimplePolygon} bertanggung jawab untuk mencari tahu apakah sebuah titik \fakesc{Point} $P$ berada di dalam \fakesc{Polygon} $A$ atau tidak. Algoritma yang digunakan pada fungsi ini adalah algoritma \textit{point inside polygon} yang dapat dilihat pada bagian \ref{sec:point-inside-polygon}. Pada desain fungsi \fakesc{InSimplePolygon} ada 3 macam keluaran yaitu $-1$ untuk menandakan bahwa \fakesc{Point} $P$ berada di dalam \fakesc{Polygon} $A$, $0$ untuk menandakan bahwa \fakesc{Point} $P$ berada di salah satu sisi \fakesc{Polygon} $A$, dan $1$ untuk menandakan bahwa \fakesc{Point} $P$ berada di luar \fakesc{Polygon} $A$. Pseudocode fungsi \fakesc{InSimplePolygon} dapat dilihat pada pseudocode \ref{psdo:fungsi-insimplepolygon}. Masukan, proses, dan keluaran dari fungsi ini tercantum pada tabel \ref{tab:fungsi-insimplepolygon}.

\begin{table}[htb]
	\Centering
	\caption{Masukan, Proses, dan Keluaran dari Fungsi \fakesc{InSimplePolygon} }
	\begin{tabular}{|p{3cm}|p{3cm}|p{3cm}|}
	\hline
	Masukan   & Proses     & Keluaran \\ \hline
	Sebuah \fakesc{Point} $A$ dan sebuah \fakesc{Polygon} $P$ yang akan dicari tahu apakah \fakesc{Point} $A$ berada di dalam \fakesc{Polygon} $P$ & Mencari tahu apakah \fakesc{Point} $A$ berada di dalam \fakesc{Polygon} $P$ &   Sebuah bilangan yang apakah \fakesc{Point} $A$ berada di dalam \fakesc{Polygon} $P$  \\ \hline
	\end{tabular}
	\label{tab:fungsi-insimplepolygon}
\end{table}

\begin{algorithm}
    \caption{Fungsi \fakesc{InSimplePolygon}}
	\label{psdo:fungsi-insimplepolygon}
    \begin{algorithmic}[1]
        \Require $P, A$
        \State $ret \leftarrow$ \fakesc{Integer}
        \For{$i \leftarrow 0$ to \fakesc{Size}($A.P$)}
            \If{$P = A.P[i]$}
                \State \Return $0$
            \EndIf
            \State $j \leftarrow A.$\fakesc{Next}($i$)
            \If{\fakesc{OnSegment}($P$, \fakesc{Segment}($A.P[i], A.P[j]$))}
                \State \Return $0$
            \EndIf
            \State $below \leftarrow (A.P[i].y < P.y)$
            \If{$below \neq (A.P[j].y<P.y)$}
                \State $o \leftarrow$ \fakesc{Orientation}($P, A.P[i], A.P[j]$)
                \If{$o = 0$}
                    \State \Return $0$
                \EndIf
                \If{$below = (o>0)$ and $below =$ \fakesc{True}}
                    \State $ret+=1$
                \Else
                    \If{$below = (o>0)$ and $below =$ \fakesc{False}}
                        \State $ret-=1$
                    \EndIf
                \EndIf
            \EndIf
        \EndFor
        \If{$ret=0$}
            \State \Return $1$
        \Else
            \State \Return $-1$
        \EndIf
	\end{algorithmic}
\end{algorithm}

\newpage
\subsection{ Desain Fungsi GetBetween}
\label{sec:fungsi-getbetween}
Fungsi \fakesc{getBetween} bertanggung jawab untuk mencari list \fakesc{Point} yang akan menggantikan \fakesc{Point} yang akan dibuang. Pseudocode fungsi \fakesc{GetBetween} dapat dilihat pada pseudocode \ref{psdo:fungsi-getbetween}. Masukan, proses, dan keluaran dari fungsi ini tercantum pada tabel \ref{tab:fungsi-getbetween}.
\begin{table}[htb]
	\Centering
	\caption{Masukan, Proses, dan Keluaran dari Fungsi \fakesc{GetBetween} }
	\begin{tabular}{|p{3cm}|p{3cm}|p{3cm}|}
	\hline
	Masukan   & Proses     & Keluaran \\ \hline
	Sebuah \fakesc{Polygon} $triangle$ yang merupakan segitiga, \fakesc{Queue} \fakesc{Point} $q$ yang merupakan pembentuk polygon luar, dan array \fakesc{Point} $trees$ yang merupakan titik yang berada di dalam polygon luar & Mencari \fakesc{Point} mana saja yang akan menggantikan \fakesc{Point} $triangle[1]$ yang akan dibuang &   Sebuah \fakesc{List} \fakesc{Point} yang berisikan daftar \fakesc{Point} yang akan menggantikan \fakesc{Point} $triangle[1]$  \\ \hline
	\end{tabular}
	\label{tab:fungsi-getbetween}
\end{table}

\begin{algorithm}
    \caption{Fungsi \fakesc{GetBetween}}
	\label{psdo:fungsi-getbetween}
    \begin{algorithmic}[1]
        \Require $triangle, q, trees$
        \State $points,pts \leftarrow$ Array \fakesc{Point}
        \While{$q$ not \fakesc{Empty}}
            \If{\fakesc{InSimplepolygon}(q.\fakesc{front}(), $triangle$) $\neq 1$}
                \State $points.$\fakesc{Insert}($q.$\fakesc{Front}())
            \EndIf
            \State $q.$\fakesc{Pop}
        \EndWhile
        \For{$pt$ in $trees$}
            \If{\fakesc{InSimplepolygon}($pt$, $triangle$) $\neq 1$}
                \State $points.$\fakesc{Insert}($pt$)
            \EndIf
        \EndFor
        \State $P \leftarrow$ \fakesc{ConvexHull}($points$)
        \State $i \leftarrow 0$
        \While{\fakesc{True}}
            \If{$P.P[i]=triangle.P[0]$}
                \If{$P.P[P.$\fakesc{Next}($i$)$]=triangle.P[2]$}
                    \State $i \leftarrow P.$\fakesc{Prev}($i$)
                    \While{$P.P[i] \neq triangle.P[2]$}
                        \State $pts.$\fakesc{Insert}($P.P[i]$)
                        \State $i = P.$\fakesc{Prev}($i$)
                    \EndWhile
                \Else
                    \State $i \leftarrow P.$\fakesc{Next}($i$)
                    \While{$P.P[i] \neq triangle.P[2]$}
                        \State $pts.$\fakesc{Insert}($P.P[i]$)
                        \State $i = P.$\fakesc{Next}($i$)
                    \EndWhile
                \EndIf
                \State \fakesc{Break}
            \EndIf
            \State $i\leftarrow P.$\fakesc{Next}($i$)
        \EndWhile
        \State \Return $pts$
	\end{algorithmic}
\end{algorithm}
\newpage
\subsection{ Desain Fungsi Solve}
\label{sec:fungsi-solve}
Fungsi \fakesc{Solve} bertanggung jawab untuk mencari \textit{relative convex polygon} yang mengelilingi semua titik yang ada di dalam polygon luar. Pseudocode fungsi \fakesc{Solve} dapat dilihat pada pseudocode \ref{psdo:fungsi-solve}. Masukan, proses, dan keluaran dari fungsi ini tercantum pada tabel \ref{tab:fungsi-solve}.
\begin{table}[htb]
	\Centering
	\caption{Masukan, Proses, dan Keluaran dari Fungsi \fakesc{Solve} }
	\begin{tabular}{|p{3cm}|p{3cm}|p{3cm}|}
	\hline
	Masukan   & Proses     & Keluaran \\ \hline
	Sebuah \fakesc{Polygon} $perimeter$ yang merupakan polygon sederhana yang merupakan polygon pembatas, dan array \fakesc{Point} $trees$ yang merupakan titik yang berada di dalam polygon pembatas & Mencari \textit{relative convex hull} dari semua titik $trees$ di dalam \fakesc{Polygon} $perimeter$  &   Sebuah \fakesc{Polygon} yang merupakan \textit{relative convex hull} dari semua titik $trees$ \\ \hline
	\end{tabular}
	\label{tab:fungsi-solve}
\end{table}


\begin{algorithm}
    \caption{Fungsi \fakesc{Solve}}
	\label{psdo:fungsi-solve}
    \begin{algorithmic}[1]
        \Require $perimeter, trees$
        \State $q \leftarrow$ \fakesc{Queue}
        \For{$pt$ in $perimeter.P$} 
        \State $q.$\fakesc{Push} $pt$
		\EndFor
		\State $bef \leftarrow perimeter.P[perimeter.$\fakesc{Size}$()-1]$
		\While{\fakesc{True}}
			\State $erased \leftarrow$ \fakesc{False}
			\State $count \leftarrow q.$\fakesc{Size}()
			\While{count--}
				\State $cur \leftarrow q.$\fakesc{Front}()
				$q.$\fakesc{Pop}()
				\If{$cur.fixed =$\fakesc{False} and (\fakesc{Fine}($q,cur$) or $cur = bef$ or $cur = q.$\fakesc{Front}())}
					\State $erased \leftarrow$ \fakesc{True}
					\State $triangle \leftarrow$ \fakesc{Polygon}
					\State $triangle.P.$\fakesc{Insert}($bef, cur, q.$\fakesc{Front}())
					\State $points \leftarrow$ \fakesc{GetBetween}($triangle,q,trees$)
					\For{$pt$ in $points$}
						\State $q.$\fakesc{Push}($pt$);
						$bef \leftarrow pt$
					\EndFor
				\Else
					\State $q.$\fakesc{Push}($cur$);
					$bef \leftarrow cur$
				\EndIf
			\EndWhile
			\If{$erased =$ \fakesc{False}}
				\fakesc{Break}
			\EndIf
		\EndWhile  
		\State $hull \leftarrow$ array of \fakesc{Point}
		\While{$q$ not empty}
			\State $hull.$\fakesc{Insert}($q.$\fakesc{Front}())
			\State $q.$\fakesc{Pop}()
		\EndWhile
		\State \Return \fakesc{Polygon}($hull$)
	\end{algorithmic}
\end{algorithm}

\newpage
\section{ Desain Pembangkit Kasus Uji}
\label{sec:pembangkit-kasus-uji}
Pada subbab ini akan dijelaskan mengenai gambaran secara umum dari pembangkit kasus uji untuk melakukan uji coba lokal pada Tugas Akhir ini.
Sistem \fakesc{GenerateTestCase} bertanggung jawab untuk membuat kasus uji berdasarkan banyaknya titik yang ingin di uji. Pseudocode fungsi \fakesc{GenerateTestCase} dapat dilihat pada pseudocode \ref{psdo:pembangkit-kasus-uji}. Masukan, proses, dan keluaran dari fungsi ini tercantum pada tabel \ref{tab:pembangkit-kasus-uji}.
\begin{table}[htb]
	\Centering
	\caption{Masukan, Proses, dan Keluaran dari Fungsi \fakesc{GenerateTestCase} }
	\begin{tabular}{|p{3cm}|p{3cm}|p{3cm}|}
	\hline
	Masukan   & Proses     & Keluaran \\ \hline
	Sebuah bilangan bulat $N$ dan $M$ yang merupakan banyaknya vertex polygon dan banyaknya titik di dalam polygon & Membangun kasus uji sebanyak masukan  &   Sebuah polygon sederhana dan sekumpulan titik acak sebanyak $N$ yang berada di dalam polygon sederhana\\ \hline
	\end{tabular}
	\label{tab:pembangkit-kasus-uji}
\end{table}


\begin{algorithm}
    \caption{Fungsi \fakesc{GenerateTestCase}}
	\label{psdo:pembangkit-kasus-uji}
    \begin{algorithmic}[1]
        \Require $N, M$
        \State $polygon \leftarrow$ \fakesc{Array}
        \State $trees \leftarrow$ \fakesc{Array}
        \For{$i \leftarrow 0$ to $N/4$} 
        	\State $polygon.$\fakesc{Push} ([$-10000+(i*20000/N), -10000$])
		\EndFor
        \For{$i \leftarrow 0$ to $N/4$} 
        	\State $polygon.$\fakesc{Push} ([10000, $-10000+(i*20000/N)$])
		\EndFor
        \For{$i \leftarrow 0$ to $N/4$} 
        	\State $polygon.$\fakesc{Push} ([$10000-(i*20000/N), 10000$])
		\EndFor
        \For{$i \leftarrow 0$ to $N-3*N/4$} 
        	\State $polygon.$\fakesc{Push} ([$10000, 10000-(i*20000/N)$])
		\EndFor
        \For{$i \leftarrow 0$ to $N-3*N/4$} 
        	\State $trees.$\fakesc{Push} ([(\fakesc{Random}() mod $20001$)-$10000$,(\fakesc{Random}() mod $20001$)-$10000$])
		\EndFor
		\State \Return (N, M, polygon, trees)
	\end{algorithmic}
\end{algorithm}